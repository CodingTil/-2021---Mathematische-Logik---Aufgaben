\documentclass[a4paper,11pt]{article}

% Layout
\usepackage[a4paper, left=3cm, right=3cm, top=2cm, bottom=3cm]{geometry} % kleinere Ränder
\usepackage{parskip}
\usepackage{rotating}

% Umlaute in der Datei erlauben, auf Deutsch umstellen
\usepackage[T1]{fontenc}
\usepackage{lmodern}
\usepackage[utf8]{inputenc}
\usepackage[english, ngerman]{babel}
%\usepackage[english]{babel} % for submissions in ENGLISH

% Mathesymbole und Ähnliches
\usepackage{amsmath}
\usepackage{mathtools}
\usepackage{amssymb}
\usepackage{microtype}
\usepackage{stmaryrd}
\usepackage{enumitem}
\usepackage{bussproofs}

% Grafiken und PDFs einfügen
\usepackage{graphicx}
\usepackage{pdfpages}

% PDF-Tools
\usepackage[hidelinks, unicode]{hyperref}

%%%%%%%%%%%%%%%%%%%%%%%%%%%%%%%%%%%%%%%%
% TIKZ-EINSTELLUNGEN FÜR SPIELE
%%%%%%%%%%%%%%%%%%%%%%%%%%%%%%%%%%%%%%%%

% Abbildungen
\usepackage{tikz}

% einige Bibliotheken
\usetikzlibrary{calc, tikzmark}
\usetikzlibrary{positioning}
\usetikzlibrary{arrows, arrows.meta}
\usetikzlibrary{decorations.markings}

% macht Linien und Pfeile dicker
\tikzset{every picture/.style={thick, >={Latex[round, length=2.5mm, width=2.5mm]}}}
\tikzset{every path/.style={shorten <= 2pt, shorten >= 2pt}}

% Styles für die Knoten der Verifiziererin und des Falsifizierers
\tikzstyle{MCverifier} = [draw, rectangle, rounded corners=1em, semithick, minimum height=2em]
\tikzstyle{MCfalsifier} = [draw, rectangle, thick, minimum height=2em]

%%%%%%%%%%%%%%%%%%%%%%%%%%%%%%%%%%%%%%%%

% Reelle, Natürliche, Ganze, Rationale Zahlen
\newcommand{\R}{\ensuremath{\mathbb{R}}}
\newcommand{\N}{\ensuremath{\mathbb{N}}}
\newcommand{\Z}{\ensuremath{\mathbb{Z}}}
\newcommand{\Q}{\ensuremath{\mathbb{Q}}}

% Fraktur für Strukturen
\newcommand{\A}{\ensuremath{\mathfrak A}}
\newcommand{\B}{\ensuremath{\mathfrak B}}
\newcommand{\I}{\ensuremath{\mathfrak I}}

% Makros für logische Operatoren
\newcommand{\xor}{\ensuremath{\oplus}} % exklusives oder
\newcommand{\impl}{\ensuremath{\rightarrow}} % logische Implikation

% Meistens ist \varphi schöner als \phi, genauso bei \theta
\renewcommand{\phi}{\varphi}
\renewcommand{\theta}{\vartheta}

% Aufzählungen anpassen (alternativ: \arabic, \alph)
\renewcommand{\labelenumi}{(\roman{enumi})}

% BITTE NICHT ÄNDERN: interne Kommandos für die Informationen (Blattnummer, Gruppe, ...)
% PLEASE DO NOT EDIT THIS SECTION

\newcommand{\printsheet}{?}
\newcommand{\sheet}[1]{%
\renewcommand{\printsheet}{#1}%
}

\newcommand{\printgroup}{?}
\newcommand{\group}[1]{%
\renewcommand{\printgroup}{#1}%
}

\newcommand{\printmembers}{}
\newcommand{\printmember}[3]{{#2} {#3} & {#1} \\}

\newcommand{\pdfmembers}{}
\newcommand{\pdfmember}[3]{{#1} {#3}, {#2}; }

\newcommand{\member}[3]{%
\expandafter\renewcommand\expandafter\printmembers\expandafter{\printmembers\printmember{#1}{#2}{#3}}%
\expandafter\renewcommand\expandafter\pdfmembers\expandafter{\pdfmembers\pdfmember{#1}{#2}{#3}}%
}

\AtBeginDocument{\hypersetup{
    pdftitle = {Übungsblatt \printsheet},
    pdfauthor = {Abgabegruppe \printgroup: \pdfmembers},
    pdfsubject = {Mathematische Logik}
}}

% <-------- HIER alle Informationen eintragen ========================
% enter your information here
\sheet{11} % Nummer des Blatts / number of exercise sheet
\group{55} % Gruppennummer der Abgabegruppe in Moodle / group number from Moodle
% Die Gruppennummer erscheint NICHT auf dem Blatt (nur in den PDF-Metadaten).
% The group number does NOT appear on the sheet (check the PDF meta data).

% alle Gruppenmitglieder in der Form \member{Matrikelnummer}{Vorname}{Nachname}
% group members are entered as \member{matriculation number}{first name}{last name}
\member{405401}{Marc}{Ludevid}
\member{405409}{Andrés}{Montoya}
\member{405959}{Til}{Mohr}
%\member{999999}{Viertes}{Mitglied}

\begin{document}

% Platz für die Punktetabelle und Kommentare
\hfill
\begin{Form}
\begin{tabular}{c}
\\
Gesamtpunkte: \\[2mm]
\TextField[name=points, width=20mm, align=1, bordercolor={0 0 0}]{} \\
\\
\end{tabular}
\end{Form}

% Kopfzeile
{\raggedright
\begin{tabular}{l}
    MaLo \\
    SS 2021 \\
    \today{} \\
\end{tabular}}
\hfill
{\Large Übungsblatt \printsheet}
\hfill
\begin{tabular}{l l}
\printmembers
\end{tabular}
\hrule


% <-------- HIER beginnt die Lösung ========================
% your SOLUTION starts here

\section*{Aufgabe 1}
E-Test

\section*{Aufgabe 2}
Wir suchen ein unendliches Axiomensystem $\Psi$, welches die Klasse $\mathcal{K}$ widerspricht. $\Psi$ soll also genau die Klasse der ungerichteten Graphen $G$ axiomatisieren, welche eine unendliche Clique enthalten. Enthält $G$ eine unendliche Clique, so enthält $G$ offensichtlich für jedes $n \in \N\setminus\{0\}$ eine Clique der Länge $n$. Wir können $\Psi$ also wie folgt aufstellen:
$$\Psi \coloneqq \{\forall x (\neg Exx), \forall x \forall y (Exy \rightarrow Eyx)\} \cup \{\psi_n \mid n \in \N\setminus\{0\}\}$$
, wobei $\psi_n \coloneqq \exists x_1 \dots \exists x_n (\bigwedge\limits_{1 \leq i < j \leq n} x_i \neq x_j \land Ex_ix_j)$ für alle $n \in \N\setminus\{0\}$ die Hilfsformel für eine Clique der Länge $n$ ist.\\

Nehmen wir nun an, es gibt ein Axiomensystem $\Phi$, welches $\mathcal{K}$ axiomatisiert. Dann ist $\Phi \cup \Psi$ unerfüllbar. Nach dem KS existiert eine endliche Teilmenge $\Theta_0 \subseteq \Phi \cup \Psi$, welches bereits unerfüllbar ist.\\
Sei $\Psi_0 \coloneqq \Theta_0 \cap \Psi$. Es existert wegen Endlichkeit ein $m \in \N\setminus\{0\}$, sodass $\psi_n \not\in \Psi_0$ für alle $n \geq m$. Es folgt $\Psi_0 \subseteq \{\forall x (\neg Exx), \forall x \forall y (Exy \rightarrow Eyx)\} \cup \{\psi_n \mid n < m\}$\\
Betrachte $\A \coloneqq (V \coloneqq \N, E \coloneqq \N \times \N)$. $\A$ ist dann also ein ungerichteter Graph, welcher eine Clique mit unendlicher Länge ist. Es gilt also $\A \in \mathcal{K}$, weshalb auch $\A \models \Phi$ gilt.\\
Jedoch gilt auch $\A \models \Psi_0$ offensichtlich. Also folgt $\A \models \Theta_0$. Jedoch soll $\Theta_0$ unerfüllbar sein.\\

Dies ist ein Widerspruch. Also ist $\mathcal{K}$ nicht axiomatisierbar.


\newpage


\section*{Aufgabe 3}
\begin{enumerate}[label=(\alph*)]
	\item	
	\item	\begin{align*}
				\Phi_b \coloneqq &\{\forall x \forall z \exists y (x+y=z),\\
					& \exists x \forall y (x+y=y),\\
					& \forall x \forall y (x+y = y+x),\\
					& \exists x_1 \dots \exists x_n ((\bigwedge_{1 \leq i<j \leq n} x_i \neq x_y) \land (\exists y \bigvee_{1 \leq i \leq n} y = x_i))\}
			\end{align*}
			??? Darf man so Endlichkeit ausdrücken?
	\item	$U$ muss hier leer sein oder einelementig sein. Angenommen $U$ ist mindestens zweielementig, aber immer noch endlich. Dann gilt ja für alle $x,y \in U$ mit $x<y$, dass ein $z$ existiert, sodass $x<z \land z<y$. Per Induktion stellt man schnell fest, dass $U$ unendlich sein muss. Dies ist ein Widerspruch.\\
			Man kann die Klasse $\mathcal{K}_c$ axiomatisieren durch: 
			$$\Phi_c \coloneqq \{\forall x \forall y (x=y)\}$$
	\item	Da $f(U) \subseteq U$, gilt auch $\vert f(U) \vert \leq \vert U \vert$. Da $f(U)$ unendlich ist, ist folglich auch $U$ unendlich.
	\item	
	\item	
	\item	Die Signatur ist mit $\tau_g \coloneqq ((R_n)_{n\in\N})$ offensichtlich abzählbar. Wegen der Definition von $R_n$ sind alle $a_S$ unterscheidbar. Man kann also von jedem $a_S$ auf ein $S \subseteq \N$ schließen (bijektiv). Deshalb gilt: $\vert A \vert = \vert \operatorname{Pot}(\N) \vert$ überabzählbar.\\
			Satz von LS$\downarrow$:\\
			Angenommen es gibt ein $\Phi_g$, welches $\mathcal{K}_g$ axiomatisiert. Da die Signatur abzählbar ist, ist $\Phi_g$ abzählbar. Nach LS$\downarrow$ ein abzählbares Modell. Jedoch gibt es in $\mathcal{K}_g$ keine endlichen Strukturen.\\
			Widerspruch. $\mathcal{K}_g$ ist nicht axiomatisierbar.
\end{enumerate}


\newpage


\section*{Aufgabe 4}
Da die Klasse der Cliquen aufzählbar ist (für jedes $n \in \mathbb{N}$ gibt es bis auf Isomorphie genau ein Element in der Klasse der Cliquen mit $n$ Knoten) kann der Algorithmus einfach alle Cliquen durchlaufen und wird auf jeden Fall die Clique $G$ finden die $\varphi$ erfüllt.\\
Zu klären ist aber noch wann dieser Algorithmus terminieren soll, wenn es solch eine Clique nicht gibt. Dafür verwenden wir den Quantorenrang des Satzes $\varphi$. Für Quantorenrang $m$ kann der Algorithmus nach $m$ Schritten aufhören, weil ...\\
Idee: Duplikatorin kann immer nachahmen solange es genügend Knoten zum auswählen gibt, weil es immer irrelevant ist welchen Knoten aus der Clique sie wählt weil alle isomorph zueinander sind.

\newpage


\section*{Aufgabe 5*}
\begin{enumerate}[label=(\alph*)]
	\item	(i) z.z.: Für alle $\mathfrak{A} \in K$ gilt: $\mathfrak{A} \in K^*$.\\
			Folgt aus Definition von $Th(K)$. $Th(K) \coloneqq \{\varphi \in FO(\tau) \mid \mathfrak{A} \models \varphi für alle \mathfrak{A} \in K\}$, also gilt für alle $\mathfrak{A} \in K$ offensichtlich auch $\mathfrak{A} \in Mod(Th(K))$.

			(ii) z.z.: Für jedes Axiomensystem $\Phi$ sodass $K \subseteq Mod(\Phi)$ gilt auch $Mod(Th(K)) \subseteq Mod(\Phi)$.\\
			Sei also $\Phi$ beliebig sodass für jedes $\mathfrak{A} \in K$ gilt: $\mathfrak{A} \models \Phi$. Dann gilt für jedes $\varphi \in Phi$: $\varphi \in Th(K)$ wegen definition von der Theorie einer Klasse. Also gilt offensichtlich auch $Mod(Th(K)) \subseteq Mod(\Phi)$
	\item	
\end{enumerate}


\end{document}
