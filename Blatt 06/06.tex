\documentclass[a4paper,11pt]{article}

% Layout
\usepackage[a4paper, left=3cm, right=3cm, top=2cm, bottom=3cm]{geometry} % kleinere Ränder
\usepackage{parskip}

% Umlaute in der Datei erlauben, auf Deutsch umstellen
\usepackage[T1]{fontenc}
\usepackage{lmodern}
\usepackage[utf8]{inputenc}
\usepackage[english, ngerman]{babel}
%\usepackage[english]{babel} % for submissions in ENGLISH

% Mathesymbole und Ähnliches
\usepackage{amsmath}
\usepackage{mathtools}
\usepackage{amssymb}
\usepackage{microtype}
\usepackage{stmaryrd}
\usepackage{enumitem}
\usepackage{bussproofs}

% Grafiken und PDFs einfügen
\usepackage{graphicx}
\usepackage{pdfpages}

% PDF-Tools
\usepackage[hidelinks, unicode]{hyperref}

% Abbildungen
\usepackage{tikz}
\usetikzlibrary{arrows,calc}

% Reelle, Natürliche, Ganze, Rationale Zahlen
\newcommand{\R}{\ensuremath{\mathbb{R}}}
\newcommand{\N}{\ensuremath{\mathbb{N}}}
\newcommand{\Z}{\ensuremath{\mathbb{Z}}}
\newcommand{\Q}{\ensuremath{\mathbb{Q}}}

% Fraktur für Strukturen
\newcommand{\A}{\ensuremath{\mathfrak A}}
\newcommand{\B}{\ensuremath{\mathfrak B}}
\newcommand{\I}{\ensuremath{\mathfrak I}}

% Makros für logische Operatoren
\newcommand{\xor}{\ensuremath{\oplus}} % exklusives oder
\newcommand{\impl}{\ensuremath{\rightarrow}} % logische Implikation

% Meistens ist \varphi schöner als \phi, genauso bei \theta
\renewcommand{\phi}{\varphi}
\renewcommand{\theta}{\vartheta}

% Aufzählungen anpassen (alternativ: \arabic, \alph)
\renewcommand{\labelenumi}{(\roman{enumi})}

% BITTE NICHT ÄNDERN: interne Kommandos für die Informationen (Blattnummer, Gruppe, ...)
% PLEASE DO NOT EDIT THIS SECTION

\newcommand{\printsheet}{?}
\newcommand{\sheet}[1]{%
\renewcommand{\printsheet}{#1}%
}

\newcommand{\printgroup}{?}
\newcommand{\group}[1]{%
\renewcommand{\printgroup}{#1}%
}

\newcommand{\printmembers}{}
\newcommand{\printmember}[3]{{#2} {#3} & {#1} \\}

\newcommand{\pdfmembers}{}
\newcommand{\pdfmember}[3]{{#1} {#3}, {#2}; }

\newcommand{\member}[3]{%
\expandafter\renewcommand\expandafter\printmembers\expandafter{\printmembers\printmember{#1}{#2}{#3}}%
\expandafter\renewcommand\expandafter\pdfmembers\expandafter{\pdfmembers\pdfmember{#1}{#2}{#3}}%
}

\AtBeginDocument{\hypersetup{
    pdftitle = {Übungsblatt \printsheet},
    pdfauthor = {Abgabegruppe \printgroup: \pdfmembers},
    pdfsubject = {Mathematische Logik}
}}

% <-------- HIER alle Informationen eintragen ========================
% enter your information here
\sheet{06} % Nummer des Blatts / number of exercise sheet
\group{55} % Gruppennummer der Abgabegruppe in Moodle / group number from Moodle
% Die Gruppennummer erscheint NICHT auf dem Blatt (nur in den PDF-Metadaten).
% The group number does NOT appear on the sheet (check the PDF meta data).

% alle Gruppenmitglieder in der Form \member{Matrikelnummer}{Vorname}{Nachname}
% group members are entered as \member{matriculation number}{first name}{last name}
\member{405401}{Marc}{Ludevid}
\member{405409}{Andrés}{Montoya}
\member{405959}{Til}{Mohr}
%\member{999999}{Viertes}{Mitglied}

\begin{document}

% Platz für die Punktetabelle und Kommentare
\hfill
\begin{Form}
\begin{tabular}{c}
\\
Gesamtpunkte: \\[2mm]
\TextField[name=points, width=20mm, align=1, bordercolor={0 0 0}]{} \\
\\
\end{tabular}
\end{Form}

% Kopfzeile
{\raggedright
\begin{tabular}{l}
    MaLo \\
    SS 2021 \\
    \today{} \\
\end{tabular}}
\hfill
{\Large Übungsblatt \printsheet}
\hfill
\begin{tabular}{l l}
\printmembers
\end{tabular}
\hrule


% <-------- HIER beginnt die Lösung ========================
% your SOLUTION starts here

\section*{Aufgabe 1}
E-Test

\section*{Aufgabe 2}
\begin{enumerate}[label=(\alph*)]
	\item	\begin{align*}
				\theta_1	&\coloneqq Qy \lor \neg \forall x (Px \impl \exists z \neg (Rfzz \land c < z)) \\
							&\equiv Qy \lor \exists x (\neg (PX \impl \exists z \neg (Rfzz \land c < z))) \\
							&\equiv Qy \lor \exists x (\neg (\neg PX \lor \exists z \neg (Rfzz \land c < z))) \\
							&\equiv Qy \lor \exists x (PX \land \forall z (Rfzz \land c < z))
			\end{align*}
	\item	\begin{align*}
				\theta_1	&\coloneqq (\neg \exists x Pffx \lor \forall x (Qx \land Rxy)) \land x < y \land Rcz \\
							&\equiv (\forall a (\neg Pffa) \lor \forall b (Qb \land Rby)) \land x < y \land Rcz  \\
							&\equiv \forall a \forall b (((\neg Pffa) \lor (Qb \land Rby)) \land x < y \land Rcz)  \\
			\end{align*}
\end{enumerate}


%\newpage


\section*{Aufgabe 3}
\begin{enumerate}[label=(\alph*)]
	\item	$$(\mathbb{N}, +, 0)$$
	\item	$$\varphi = \forall x (x + f(x) = 0)$$ $\varphi$ und $\psi$ sind nicht logisch äquivalent.
	\item	$$\mathfrak{B} \coloneqq (\mathbb{Z}, +, -, 0)$$ Dann kann man $f$ wählen als $f(x) = 0-x$. Dann gilt $\mathfrak{B} \models \varphi$.
	\item	Nein gibt es nicht. Jede Substruktur $\mathfrak{C}$ von $\mathfrak{B}$ muss $\{+,-,0\}$ abgeschlossen sein. Damit muss $0$ immer im Universum enthalten sein. Es gilt jedoch $(\{0\}, +, -, 0) \models \psi$ (da $0+0=0$). Fügt man nur schon ein weiteres Element dem Universum hinzu, muss aufgrund der $\{-\}$-Abgeschlossenheit auch das additive Inverse hinzugefügt werden. Damit gilt für jede Substruktur $\psi$
\end{enumerate}


\newpage


\section*{Aufgabe 4}
\begin{enumerate}[label=(\alph*)]
	\item	\begin{enumerate}[label=(\roman*)]
				\item	\begin{align*}
							\Phi_i \coloneqq \{	& \forall x \forall y (x \circ y = y \circ x), 	& \text{kommutativ} \\
												& \forall x \forall y \forall z (x \circ (y \circ z) = (x \circ y) \circ z), 	& \text{assoziativ} \\
												& \forall x (x \circ e = x), 	& \text{e ist neutrales Element} \\
												& \forall x \exists y (x \circ y = e), 	& \text{inverses Element} \\
												& \exists x_1 \dots \exists x_6 (\bigwedge_{\substack{1 \leq i,j \leq 6 \\ i \neq j}} x_i \neq x_j),	& \text{mindestens 6 Elemente} \\
												& \forall x_1 \dots \forall x_{10} ((\bigwedge_{\substack{1 \leq i,j \leq 9 \\ i \neq j}} x_i \neq x_j) \impl (\bigvee_{1 \leq i \leq 9} x_i = x_{10}))\} 	& \text{nicht mindestens 10 Elemente}
						\end{align*}
				\item	\begin{align*}
							\Phi_{ii} \coloneqq \{	& \forall x (x < x), 	& \text{reflexiv} \\
													& \forall x \forall y ((x < y \land y < x) \impl x = y), 	& \text{asymmetrie} \\
													& \forall x \forall y \forall z ((x < y \land y < z) \impl x < z)\} 	& \text{transitiv}
						\end{align*}
				\item	\begin{align*}
							\Phi_{iii} \coloneqq \{	& \forall x \forall y (f(x) = f(y) \impl x = y),	& \text{injektiv} \\
													& \forall x \exists y (y = f(x) \land Oy), 	& f(U) \subseteq O \\
													& \forall y \exists x (Oy \impl f(x) = y), 	& O \subseteq f(U) \\
													& \neg Os\} 	& s \not\in O
						\end{align*}
				\item	\begin{align*}
							\Phi_{iv} \coloneqq \{	& \forall x \forall y (Exy \impl Eyx),	& \text{Beidseitige Kantenrichtung} \\
										& \neg Ecd, 	& \text{c und d nicht verbunden} \\
										& \forall x_1 (\neg (Ecx_1 \land Ex_1d)),	& \text{c und d nicht über 1 Knoten verbunden} \\
										& \forall x_1 \forall x_2 (\neg (Ecx_1 \land Ex_1x_2 \land Ex_2d)),	& \text{c und d nicht über 2 Knoten verbunden} \\
										& \dots \}	& \dots
						\end{align*}
			\end{enumerate}
	\item	$\mathfrak{A} = (\mathbb{N}, f, \mathbb{N}\setminus \{0\}, 0)$ mit $f: \mathbb{N} \rightarrow \mathbb{N}\setminus \{0\}, x \mapsto x + 1$.\\
			Jedes Modell muss unendlich sein, da $s \not\in O = f(U)$. Demnach ist $\vert U \vert \geq \vert O \vert + 1$. Dies ist nur erfüllt, wenn sowohl $U$ als auch $O$ unendlich sein.
	\item	Bauch sagt ja.
\end{enumerate}


\newpage


\section*{Aufgabe 5}
\begin{enumerate}[label=(\alph*)]
	\item	Sei $\varphi = x \lor y$, $\psi = \neg x \lor y$ und $\Phi = \{y\}$. Dann ist offensichtlich $\varphi \not\equiv \psi$, jedoch gilt $\Phi \models \varphi \leftrightarrow \psi$.
	\item	
	\item	
\end{enumerate}


\end{document}
