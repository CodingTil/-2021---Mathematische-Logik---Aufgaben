\documentclass[a4paper,11pt]{article}

% Layout
\usepackage[a4paper, left=3cm, right=3cm, top=2cm, bottom=3cm]{geometry} % kleinere Ränder
\usepackage{parskip}

% Umlaute in der Datei erlauben, auf Deutsch umstellen
\usepackage[T1]{fontenc}
\usepackage{lmodern}
\usepackage[utf8]{inputenc}
\usepackage[english, ngerman]{babel}
%\usepackage[english]{babel} % for submissions in ENGLISH

% Mathesymbole und Ähnliches
\usepackage{amsmath}
\usepackage{mathtools}
\usepackage{amssymb}
\usepackage{microtype}
\usepackage{stmaryrd}
\usepackage{enumitem}

% Grafiken und PDFs einfügen
\usepackage{graphicx}
\usepackage{pdfpages}

% PDF-Tools
\usepackage[hidelinks, unicode]{hyperref}

% Abbildungen
\usepackage{tikz}
\usetikzlibrary{arrows,calc}

% Reelle, Natürliche, Ganze, Rationale Zahlen
\newcommand{\R}{\ensuremath{\mathbb{R}}}
\newcommand{\N}{\ensuremath{\mathbb{N}}}
\newcommand{\Z}{\ensuremath{\mathbb{Z}}}
\newcommand{\Q}{\ensuremath{\mathbb{Q}}}

% Fraktur für Strukturen
\newcommand{\A}{\ensuremath{\mathfrak A}}
\newcommand{\B}{\ensuremath{\mathfrak B}}
\newcommand{\I}{\ensuremath{\mathfrak I}}

% Makros für logische Operatoren
\newcommand{\xor}{\ensuremath{\oplus}} % exklusives oder
\newcommand{\impl}{\ensuremath{\rightarrow}} % logische Implikation

% Meistens ist \varphi schöner als \phi, genauso bei \theta
\renewcommand{\phi}{\varphi}
\renewcommand{\theta}{\vartheta}

% Aufzählungen anpassen (alternativ: \arabic, \alph)
\renewcommand{\labelenumi}{(\roman{enumi})}

% BITTE NICHT ÄNDERN: interne Kommandos für die Informationen (Blattnummer, Gruppe, ...)
% PLEASE DO NOT EDIT THIS SECTION

\newcommand{\printsheet}{?}
\newcommand{\sheet}[1]{%
\renewcommand{\printsheet}{#1}%
}

\newcommand{\printgroup}{?}
\newcommand{\group}[1]{%
\renewcommand{\printgroup}{#1}%
}

\newcommand{\printmembers}{}
\newcommand{\printmember}[3]{{#2} {#3} & {#1} \\}

\newcommand{\pdfmembers}{}
\newcommand{\pdfmember}[3]{{#1} {#3}, {#2}; }

\newcommand{\member}[3]{%
\expandafter\renewcommand\expandafter\printmembers\expandafter{\printmembers\printmember{#1}{#2}{#3}}%
\expandafter\renewcommand\expandafter\pdfmembers\expandafter{\pdfmembers\pdfmember{#1}{#2}{#3}}%
}

\AtBeginDocument{\hypersetup{
    pdftitle = {Übungsblatt \printsheet},
    pdfauthor = {Abgabegruppe \printgroup: \pdfmembers},
    pdfsubject = {Mathematische Logik}
}}

% <-------- HIER alle Informationen eintragen ========================
% enter your information here
\sheet{03} % Nummer des Blatts / number of exercise sheet
\group{55} % Gruppennummer der Abgabegruppe in Moodle / group number from Moodle
% Die Gruppennummer erscheint NICHT auf dem Blatt (nur in den PDF-Metadaten).
% The group number does NOT appear on the sheet (check the PDF meta data).

% alle Gruppenmitglieder in der Form \member{Matrikelnummer}{Vorname}{Nachname}
% group members are entered as \member{matriculation number}{first name}{last name}
\member{405401}{Marc}{Ludevid}
\member{405409}{Andrés}{Montoya}
\member{405959}{Til}{Mohr}
%\member{999999}{Viertes}{Mitglied}

\begin{document}

% Platz für die Punktetabelle und Kommentare
\hfill
\begin{Form}
\begin{tabular}{c}
\\
Gesamtpunkte: \\[2mm]
\TextField[name=points, width=20mm, align=1, bordercolor={0 0 0}]{} \\
\\
\end{tabular}
\end{Form}

% Kopfzeile
{\raggedright
\begin{tabular}{l}
    MaLo \\
    SS 2021 \\
    \today{} \\
\end{tabular}}
\hfill
{\Large Übungsblatt \printsheet}
\hfill
\begin{tabular}{l l}
\printmembers
\end{tabular}
\hrule


% <-------- HIER beginnt die Lösung ========================
% your SOLUTION starts here

\section*{Aufgabe 1}
E-Test

\section*{Aufgabe 2}
\begin{enumerate}[label=(\alph*)]
	\item Um die gegebene Folgerungsbeziehung zu beweisen zeigen wir, dass $\{ A \lor E \lor D, (D	\rightarrow B) \land \neg (C \rightarrow (D \land B)), \neg (A \lor (E \land C)) \}$ unerfüllbar ist.

		Dazu konstruieren wir zunächst eine geeignete Klauselmenge K:

		$K = \{ \{ A \lor E \lor D \}, \{ D \rightarrow B \}, \{ \neg (C \rightarrow (D \land B))\}, \{ \neg (A \lor (E \land C))\}\}$

		Dies kann durch mehereres Vereinfach auf folgende Klauselmenge vereinfacht werden:

		$K = \{ \{ A \lor E \lor D \}, \{ \neg D \lor B \}, \{ C \}, \{ \neg D \lor \neg B \}, \{ \neg A \}, \{ \neg E \lor \neg C\} \}$

		Nun wenden wir das Resulutionskalkül an und erhalten:

		$\{ \{ E \lor D \}, \{ \neg D \}, \{ \neg E \} \}$ und durch erneutes anwenden:

		$\{ \{ D \}, \{ \neg D \} \}$ und schliesslich:

		$\{ \square \}$

		Somit ist die Folgerungsbeziehung bewiesen.
	
	\item	\begin{enumerate}[label={i$=$\arabic*:}, start=0]
				\item	$\{\{\neg A, B, D\}, \{\neg B, \neg D\}, \{A\}, \{\neg E, D\}, \{C\}, \{C, \neg B\}, \{E\}\}$
				\item	$\{\{B,D\}, \{\neg A,C,D\}, \{\neg B, \neg E \}, \{D\}\}$
				\item	$\{\{D,\neg E\}, \{\neg B\}\}$
				\item	$\{\}$
			\end{enumerate}
			Da bei der dritten Iteration es keine Veränderung mehr gibt gilt: $\text{PRes}^*(K) = \text{PRes}_2(K)$. Da $\square \not\in \text{PRes}^*$ und der Algorithmus vollständig ist, ist $K$ erfüllbar.
\end{enumerate}



\section*{Aufgabe 3}
\begin{enumerate}[label=(\alph*)]
	\item z.z. bereinigte Resolutionskalkül ist korrekt, also für $K$ Klauselmenge, $C_1$ und $C_2$ aus $K$ und $C$ Resolvente aus $C_1$ und $C_2$	

		Es gibt 2 Fälle:
		\begin{itemize}
			\item $C$ ist nicht tautologisch und es gibt keine Klausel $C'$ in $K$ sodass $C'$ eine Untermenge von $C$ ist:

				Dieser Fall ist analog zum normalen Resoltionskalkül und somit gilt, dass $K \cup {C}$ equivalent zu $K$ ist.
			\item Andernfalls:

					In diesem Fall wird $C$ nicht zur Klauselmenge hinzugefügt und somit gilt trivialerweise, dass $K$ nach dem Resolutionsschritt equivalent zu $K$ vor dem Resoltuionsschritt ist

		\end{itemize}

		Wenn man durch wiederholtes Anwenden der Resolutionsregel eine Klauselmenge $K'$ erhält, welche die leere Menge enthält, dann ist $K$ unerfüllbar, da $K$ zu $K'$ equivalent ist.

	\item z.z. bereinigtes Resultionskalkül ist vollständig.

		Dazu zeigen wir, dass wenn leere Menge in $Res_k(K)$ ist, dann auch leere Menge in $PRes_k(K)$ ist. Da das normale Resolutionskalkül vollständig ist, muss dann auch das bereinigte Resolutionskalkül vollständig sein.

		Dazu beweisen wir per Induktion, dass für alle $i$ gilt: Für alle Klauseln $C \in Res_i(K)$ gilt: Entweder ist $C$ tautologisch oder es gibt $C' \in PRes_i(K)$ sodass $C'$ eine Untermenge von $C$ ist.
		Wenn für alle $i$ diese Aussage gilt, dann gilt offensichtlich auch, dass falls die leere Menge in $Res_k(K)$ ist, die leere Menge auch in $PRes_k(K)$ ist, da die leere Menge nicht tautologisch ist und somit es ein $C' \in PRes_k(K)$ geben muss dass eine Untermenge von der leeren Menge ist. Dies kann nur die leere Menge sein.

		Induktionsanfang: $(i=0)$: $PRes_0(K) = Res_0(K)$ also gilt die Aussage trivialerweise

		Induktionsschritt $(i \Rightarrow i+1)$:

			Da für alle $C \in Res_i(K)$ gilt, dass diese entweder tautologisch sind oder es ein $C' \in PRes_i(K)$ gibt welches eine Untermenge von $C$ ist müssen wir die Aussage ausschliesslich für alle $C \in Res_{i+1}(K)\backslash Res_i(K)$ prüfen. Für alle diese $C$ gilt:

			$C$ ist Resolvente aus $C_1, C_2 \in Res_i(K)$. Also sind jeweils $C_1$ und $C_2$ entweder tautologisch oder es exitieren entsprechende $C'_1, C'_2 \in PRes_i(K)$ sodass $C'_1$ eine Untermenge von $C_1$ ist und $C'_2$ eine Untermenge von $C_2$ ist.

			Wir unterscheiden 2 Fälle:
			\begin{itemize}
				\item $C$ ist tautologisch, dann gilt die Aussage trivialerweise.
				\item $C$ ist nicht tautologisch. Dafür kann man ebenfalls 4 Fälle unterscheiden:
					\begin{itemize}
						\item $C_1$ ist tautologisch, also $C_1$ ist Obermenge von $\{X, \bar{X}\}$ und $C_2$ nicht. Dann kann man entweder über $X$ resolvieren, wodurch das entstehende $C$ eine Obermenge von $C_2$ ist, dementsprechend gibt es in $PRes_{i+1}(K)$ eine Klausel, nämlich $C'_2$ die eine Untermenge von $C$ ist, oder man resolviert nicht über $X$, dann ist $C$ ebenfalls eine tautologische Klausel und die Aussage gilt ebenfalls.
						\item $C_2$ ist tautlogisch und $C_1$ nicht. Analog zu Fall 1.
						\item $C_1$ und $C_2$ sind tautologisch. Dann ist $C$ auch eine tautologische Klausel und die Aussage gilt.
						\item Sowohl $C_1$ als auch $C_2$ sind nicht tautologisch. Dann gibt es in $PRes_i(K)$ Klauseln $C'_1$ und $C'_2$ welche beide jeweils Untermengen von $C_1$ und $C_2$ sind. Die Resolvente $C'$ aus $C'_1$ und $C'_2$ ist eine Untermenge von $C$, da jedes Literal in $C'$ entweder aus $C'_1$ oder $C'_2$ kommt, somit auch auf jeden Fall in $C_1$ oder $C_2$ vorhanden sind und folglich auch in $C$. Da $C$ nicht tautologisch ist, ist es $C'$ auch nicht. Wenn es eine Klausel $C'' \in PRes_i(K)$ gibt sodass $C''$ eine Untermenge von $C'$ ist, dann gibt es auch in $PRes_{i+1}(K)$ eine Klausel die eine Untermenge von $C$ ist, nämlich $C''$. Wenn es so ein $C''$ nicht gibt, dann wird $C' \in PRes_{i+1}(K)$ sein und somit gibt es ebenfalls eine Klausel die eine Untermenge von $C$ ist.
					\end{itemize}
			\end{itemize}

			Somit ist per Induktion bewiesen, dass das bereinigte Resolutionskalkül vollständig ist.


\end{enumerate}





\section*{Aufgabe 4}
\begin{enumerate}[label=(\alph*)]
	\item Für $\Phi$ endlich ist die Aussage trivial mit $\Phi_0 = \Phi$

		Für $\Phi$ unendlich:

		$\Phi$ abhängig ist nach definition genau dann der Fall wenn ein $\varphi \in \Phi$ existiert sodass gilt: $\Phi \backslash \{ \varphi \} \models \varphi$.
		Wir können nun den Kompaktheitssatz anwenden und erhalten folgende equivalente Aussage: es ex. ein $\varphi \in \Phi$ und eine endl. Teilmenge $\Phi'_0$ von $\Phi \backslash \{ \varphi \}$ sodass gilt: $\Phi'_0 \models \varphi$.
		Nun definieren wir $\Phi_0 = \Phi'_0 \cup \{\varphi \}$ und stellen fest dass $\Phi'_0 \models \varphi$ genau dann gilt wenn $\Phi_0$ abhängig ist.
		Also ist $\Phi$ genau dann abhänig wenn es ein endliches $\Phi_0$ gibt dass ebenfalls abhänig ist.
			
	\item	
\end{enumerate}




\section*{Aufgabe 5}
Beweis per Kompaktheitssatz:

Sei $X_A$ für $A \in \mathcal{P}(\mathbb{N})$ genau dann $1$ wenn $A \in M$.

Damit die Bedingungen aus der Aufgabenstellung erfüllt sind muss für jedes $A \in \mathcal{P}(\mathbb{N})$ gelten, dass nach dem hinzufügen von einem $c \in \mathbb{N}$ und $c \not \in A$ gilt: $X_A = \neg X_{A \cup \{c\}}$

Somit ist $\Phi = \{X_A = \neg X_{A \cup \{c\}}$ für alle $A \in \mathcal{P}(\mathbb{N})$ und für alle $c \in \mathbb{N}$ mit $c \not \in A\}$

Sei $\Phi_0$ eine beliebige endliche Untermenge von $\Phi$. z.z.: $\Phi_0$ ist erfüllbar.

Sei $\mathfrak{I}(X_A)$ genau dann $1$ wenn $|A|$ ungerade ist. $\mathfrak{I}$ erfüllt dann $\Phi_0$ da $||A| - |A \cup \{c\}|| = 1$ für alle $A$ und für alle $c \in \mathbb{N}$ mit $c \not \in A$. Deswegen ist $\mathfrak{I}(X_A) = 1 \oplus \mathfrak{I}(X_{A \cup \{c\}}) = 1$. Somit ist jede Aussagenlogische Formel in $\Phi_0$ trivialerweise erfüllt.

Da jede endliche Teilmenge von $\Phi$ erfüllbar ist, ist $\Phi$ ebenfalls erfüllbar und somit ist es möglich alle Teilmengen der natürlichen Zahlen in zwei disjunkte Gruppen $M$ und $N$ zu teilen, sodass keine zwei benachbarten Mengen in der selben Gruppe sind.

\end{document}
