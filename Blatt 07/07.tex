\documentclass[a4paper,11pt]{article}

% Layout
\usepackage[a4paper, left=3cm, right=3cm, top=2cm, bottom=3cm]{geometry} % kleinere Ränder
\usepackage{parskip}

% Umlaute in der Datei erlauben, auf Deutsch umstellen
\usepackage[T1]{fontenc}
\usepackage{lmodern}
\usepackage[utf8]{inputenc}
\usepackage[english, ngerman]{babel}
%\usepackage[english]{babel} % for submissions in ENGLISH

% Mathesymbole und Ähnliches
\usepackage{amsmath}
\usepackage{mathtools}
\usepackage{amssymb}
\usepackage{microtype}
\usepackage{stmaryrd}
\usepackage{enumitem}
\usepackage{bussproofs}

% Grafiken und PDFs einfügen
\usepackage{graphicx}
\usepackage{pdfpages}

% PDF-Tools
\usepackage[hidelinks, unicode]{hyperref}

%%%%%%%%%%%%%%%%%%%%%%%%%%%%%%%%%%%%%%%%
% TIKZ-EINSTELLUNGEN FÜR SPIELE
%%%%%%%%%%%%%%%%%%%%%%%%%%%%%%%%%%%%%%%%

% Abbildungen
\usepackage{tikz}

% einige Bibliotheken
\usetikzlibrary{calc, tikzmark}
\usetikzlibrary{positioning}
\usetikzlibrary{arrows, arrows.meta}
\usetikzlibrary{decorations.markings}

% macht Linien und Pfeile dicker
\tikzset{every picture/.style={thick, >={Latex[round, length=2.5mm, width=2.5mm]}}}
\tikzset{every path/.style={shorten <= 2pt, shorten >= 2pt}}

% Styles für die Knoten der Verifiziererin und des Falsifizierers
\tikzstyle{MCverifier} = [draw, rectangle, rounded corners=1em, semithick, minimum height=2em]
\tikzstyle{MCfalsifier} = [draw, rectangle, thick, minimum height=2em]

%%%%%%%%%%%%%%%%%%%%%%%%%%%%%%%%%%%%%%%%

% Reelle, Natürliche, Ganze, Rationale Zahlen
\newcommand{\R}{\ensuremath{\mathbb{R}}}
\newcommand{\N}{\ensuremath{\mathbb{N}}}
\newcommand{\Z}{\ensuremath{\mathbb{Z}}}
\newcommand{\Q}{\ensuremath{\mathbb{Q}}}

% Fraktur für Strukturen
\newcommand{\A}{\ensuremath{\mathfrak A}}
\newcommand{\B}{\ensuremath{\mathfrak B}}
\newcommand{\I}{\ensuremath{\mathfrak I}}

% Makros für logische Operatoren
\newcommand{\xor}{\ensuremath{\oplus}} % exklusives oder
\newcommand{\impl}{\ensuremath{\rightarrow}} % logische Implikation

% Meistens ist \varphi schöner als \phi, genauso bei \theta
\renewcommand{\phi}{\varphi}
\renewcommand{\theta}{\vartheta}

% Aufzählungen anpassen (alternativ: \arabic, \alph)
\renewcommand{\labelenumi}{(\roman{enumi})}

% BITTE NICHT ÄNDERN: interne Kommandos für die Informationen (Blattnummer, Gruppe, ...)
% PLEASE DO NOT EDIT THIS SECTION

\newcommand{\printsheet}{?}
\newcommand{\sheet}[1]{%
\renewcommand{\printsheet}{#1}%
}

\newcommand{\printgroup}{?}
\newcommand{\group}[1]{%
\renewcommand{\printgroup}{#1}%
}

\newcommand{\printmembers}{}
\newcommand{\printmember}[3]{{#2} {#3} & {#1} \\}

\newcommand{\pdfmembers}{}
\newcommand{\pdfmember}[3]{{#1} {#3}, {#2}; }

\newcommand{\member}[3]{%
\expandafter\renewcommand\expandafter\printmembers\expandafter{\printmembers\printmember{#1}{#2}{#3}}%
\expandafter\renewcommand\expandafter\pdfmembers\expandafter{\pdfmembers\pdfmember{#1}{#2}{#3}}%
}

\AtBeginDocument{\hypersetup{
    pdftitle = {Übungsblatt \printsheet},
    pdfauthor = {Abgabegruppe \printgroup: \pdfmembers},
    pdfsubject = {Mathematische Logik}
}}

% <-------- HIER alle Informationen eintragen ========================
% enter your information here
\sheet{07} % Nummer des Blatts / number of exercise sheet
\group{55} % Gruppennummer der Abgabegruppe in Moodle / group number from Moodle
% Die Gruppennummer erscheint NICHT auf dem Blatt (nur in den PDF-Metadaten).
% The group number does NOT appear on the sheet (check the PDF meta data).

% alle Gruppenmitglieder in der Form \member{Matrikelnummer}{Vorname}{Nachname}
% group members are entered as \member{matriculation number}{first name}{last name}
\member{405401}{Marc}{Ludevid}
\member{405409}{Andrés}{Montoya}
\member{405959}{Til}{Mohr}
%\member{999999}{Viertes}{Mitglied}

\begin{document}

% Platz für die Punktetabelle und Kommentare
\hfill
\begin{Form}
\begin{tabular}{c}
\\
Gesamtpunkte: \\[2mm]
\TextField[name=points, width=20mm, align=1, bordercolor={0 0 0}]{} \\
\\
\end{tabular}
\end{Form}

% Kopfzeile
{\raggedright
\begin{tabular}{l}
    MaLo \\
    SS 2021 \\
    \today{} \\
\end{tabular}}
\hfill
{\Large Übungsblatt \printsheet}
\hfill
\begin{tabular}{l l}
\printmembers
\end{tabular}
\hrule


% <-------- HIER beginnt die Lösung ========================
% your SOLUTION starts here

\section*{Aufgabe 1}
E-Test

\section*{Aufgabe 2}
\begin{enumerate}[label=(\alph*)]
	\item	\begin{align*}
				W_0 &= \{7,9\} \\
				W_1 &= \{0,1,2,3,4,5\}
			\end{align*}
			\begin{center}
			\begin{tabular}{c|c|c}
				Position	& Spieler $0$	& Spieler $1$	\\
				\hline
				0 & - & 0 \\
				1 & - & 1 \\
				2 & - & 2 \\
				3 & - & 4 \\
				4 & - & 3 \\
				5 & - & 0 \\
				6 & - & - \\
				7 & 1 & - \\
				8 & - & - \\
				9 & 0 & - \\
			\end{tabular}
			\end{center}
	\item $\mathcal{G}$ ist nicht fundiert weil es im Graphen einen Zykel gibt (Knoten 6 und 8).\\
		$\mathcal{G}$ ist ebenfalls nicht determiniert weil der Knoten 6 in keiner Gewinnregion ist und die Vereinigungsmenge von den Gewinnregionen nicht alle Knoten enthält. Dass Knoten 6 in keiner Gewinnregion ist liegt daran, dass im Knoten 6 Spieler 0 nach Knoten 8 gehen würde. Dort würde jedoch Spieler 1 wieder nach 6 zurückwollen, da Knoten 7 in der Gewinnregion von Spieler 0 liegt.
	\item Spieler 0 wird einen der Knoten aus seiner Gewinnmenge wählen wollen. Somit muss Spieler 1 versuchen diese zu leeren. Wenn Spieler 1 nicht den Knoten 9 entfernt, dann wählt Spieler 0 Knoten 9 und hat somit direkt gewonnen. Also muss Spieler 1 Knoten 9 entfernen. Danmit ist auch Knoten 7 kein Knoten der Gewinnmenge mehr, denn da Knoten 9 nicht mehr zur Verfügung steht muss in diesem Knoten Spieler 0 die Kante nach Knoten 5 wählen welches ein Siegesknoten von Spieler 1 ist.
\end{enumerate}


\newpage


\section*{Aufgabe 3}
\begin{enumerate}[label=(\alph*)]
	\item	\begin{enumerate}
				\item[{$v_s \coloneqq 2$}]	\begin{align*}
												f_0: 1 \mapsto 0, 3 \mapsto 2, 6 \mapsto 8, 7 \mapsto 9 \\
												f_1: 2 \mapsto 1, 4 \mapsto 2, 8 \mapsto 6
											\end{align*}
											Spieler $1$ gewinnt.
				\item[{$v_s \coloneqq 2$}]	\begin{align*}
												f_0: 1 \mapsto 0, 3 \mapsto 2, 6 \mapsto 8, 7 \mapsto 9 \\
												f_1: 2 \mapsto 1, 4 \mapsto 2, 8 \mapsto 6
											\end{align*}
											Unentschieden. Spieler $0$ muss zu $8$ gehen, da sonst verloren. Spieler $8$ muss zurück zu $6$ gehen, da sonst verloren (Spieler $0$ kann sonst von $7$ zu $9$).
			\end{enumerate}
	\item Um zu beweisen, dass jeder abgeschnittenes Spiel für jeden Startknoten entweder eine Gewinnstrategie für einen der beiden Spieler gibt oder dass beide Spieler ein Remis erzwingen können unterscheiden wir für alle Startknoten zwei Fälle:
		\begin{itemize}
			\item Fall $v_s \in V$ hat Gewinnstrategie für einen der Spieler: Aussage gilt trivialerweise.

			\item Fall $v_s \in V$ hat keine Gewinnstrategie für beide Spieler: z.z. Es gibt eine Strategie für beide Spieler um ein Remis zu erzwingen. Dazu konstruieren wir zwei neue zykelfreie Graphen in denen ein Remis jeweils als Gewinn für einen der beiden Spieler gilt. Diese neue Graphen entstehen durch eine Tiefensuche über den ursprünglichen Graphen. Man startet beim Startknoten $v_s$, fügt diesen zu den neuen Graphen hinzu und sucht alle Nachfolgeknoten $v_i \in v_sE$. Für jeden dieser Nachfolgeknoten $v_i \in v_sE$ fügt man einen Knoten $v_{s,i}$ in die neuen Graphen hinzu. Dabei ist $v_{s,i} \in V_0$ des neuen Graphens falls $v_i \in V_0$ im alten Graphen galt. Ebenso gilt $v_{s,i} \in V_1$ falls $v_i \in V_1$. An dem Index der neuen Knoten kann man erkennen welche Knoten des ursprünglichen Graphens durchlaufen wurden. Somit ist ein Knoten genau dann ein Remis wenn ein Indize doppelt vorkommt. In diesen Fällen liegt $v_{s,i} \in V_1$ im ersten neuen Graphen und $v_{s,i} \in V_0$ im zweiten neuen Graphen. Dies ist hier natürlich nur möglich wenn $s = i$ also wenn der Knoten $v_s$ eine Kante zu sich selber hat. Für alle anderen Knoten wird die Tiefensuche gleichermassen fortgeführt. Zusammenfassend haben wir zwei zykelfreien Graphen in dem jeweils der Spieler 0 oder 1 bei Remis gewinnt. Bis auf das Verhalten bei Remis verhalten sich diese zwei Graphen offensichtlich genau gleich wie der ursprüngliche Graph, denn es werden alle möglichen Spiele aufgelistet die dort möglich gewesen wären, wobei das Gewinnverhalten bis auf das Remis beibehalten bleibt.\\
				Betrachtet man nun den ersten neuen Graphen fallen zwei neue Eigenschaften auf:
				\begin{itemize}
					\item Da der Graph zykelfrei ist, ist er fundiert und somit determiniert. Dementsprechen gilt: $W_0 \cup W_1 = V$.
					\item Weil es für $v_s$ keine Gewinnstrategie für Spieler 1 im ursprünglichen Graphen gab gilt auch in diesem neuen Graphen $v_s \not \in W_1$, denn es wurden keine neuen Knoten hinzugefügt die zu einem Gewinn von Spieler 1 führen.
				\end{itemize}
				Daraus folgt dass $v_s \in W_0$ und da $v_s \not \in W_0$ im ursprünglichen Graphen galt, muss die Gewinnstrategie zu einem Remis führen. Somit hat Spieler 0 eine Spielstrategie das ein Remis erzwingt.\\
				Das selbe Argument gilt für Spieler 1 und dem zweiten neuen Graphen. Deswegen hat auch Spieler 1 eine Spielstrategie um ein Remis zu erzwingen.
		\end{itemize}
	Da die Aussge für beide Fälle gilt, ist die Aussage bewiesen.
\end{enumerate}


\newpage


\section*{Aufgabe 4}
$\A \models \psi$ kann als folgendes Auswertungsspiel $MC(\A, \psi)$ modeliert werden:

\begin{center}
	\begin{tikzpicture}
		\node[MCfalsifier] (psi) at (0,0) {$\psi$};
		\node[MCverifier] (and1) at (-4, -1) {$\exists x(x \sim x)$};
		\node[MCfalsifier] (and2) at (4, -1) {$\forall y (\neg Py \rightarrow Qy)$};
		\node[MCfalsifier] (1-sim-1) at (-6, -3) {$1 \sim 1$};
		\node[MCverifier] (2-sim-2) at (-4, -3) {$2 \sim 2$};
		\node[MCfalsifier] (3-sim-3) at (-2, -3) {$3 \sim 3$};
		\node[MCfalsifier] (forall-1) at (1, -3) {$\neg P1 \rightarrow Q1$};
		\node[MCfalsifier] (forall-2) at (4, -3) {$\neg P2 \rightarrow Q2$};
		\node[MCverifier] (forall-3) at (7, -3) {$\neg P3 \rightarrow Q3$};
		\draw[->] (psi) to (and1);
		\draw[->, blue] (psi) to (and2);
		\draw[->] (and1) to (1-sim-1);
		\draw[->] (and1) to (2-sim-2);
		\draw[->] (and1) to (3-sim-3);
		\draw[->] (and2) to (forall-1);
		\draw[->] (and2) to (forall-2);
		\draw[->, blue] (and2) to (forall-3);
	\end{tikzpicture}
\end{center}

Der Falsifizierer hat eine Gewinnstrategie von der Ausgangsposition aus: siehe blaue Pfeile. Also gilt $\A \models \psi$ nicht.

\newpage


\section*{Aufgabe 5}
\begin{enumerate}[label=(\alph*)]
	\item	\begin{enumerate}[label=(\roman*)]
			\item	Sei $\pi$ die Abbildung die jede Zahl $n \in \N$ in seine Primfaktoren zerlegt, jedes Vorkommen von 2 durch eine 3 ersetzt und jedes Vorkommen von einer 3 durch eine 2 ersetzt und diese Faktoren dann wieder aufmultipliziert. Diese Abbildung ist bijektiv, weil $a \not = b \Rightarrow \pi(a) \not = \pi(b)$, denn wegen $a \not = b$ sind auch die Primfaktorenzerlegungen unterschiedlich und auch nach dem Vertauschen von 2 und 3 diese unterschiedlich sein müssen und somit $\pi(a) \not = \pi(b)$ gelten muss, und $\forall y \exists x (y = \pi(x))$ indem man $x = \pi(y)$ wählt, da $\pi(\pi(x)) = x$ für alle $x \in \N$.\\
				Nun muss noch gezeigt werden, dass für alle $a_1, a_2 \in \N$ gilt: $\pi(a_1 \cdot a_2) = \pi(a_1) \cdot \pi(a_2)$. Das ist jedoch offensichtlich, da bei Multiplikation die Primfaktoren beider Zahlen erhalten bleiben und es somit irrelevant ist ob die Vertauschung der Primfaktoren vor oder nach der Multiplikation stattfindet.\\
				Desweiteren gilt offensichtlich $\forall x (\mathbb{P}x \leftrightarrow \mathbb{P}\pi(x))$. Ist nämlich $x \in \mathbb{P}$, so besteht $x$ per Definition nur aus einem Primfaktor. Nach der Konstruktion von $\pi$ besteht $\pi(x)$ dann aber auch nur aus einem Primfaktor (womöglich aber einen von dem von $x$ verschiedenen). Deshalb ist $\pi(x) \in \mathbb{P}$. Ist $x \not\in \mathbb{P}$, so besteht $x$ aus mindestens zwei Primfaktoren. Deshalb besteht $\pi(x)$ auch aus mindestens zwei Primfaktoren und ist somit auch nicht in $\mathbb{P}$ enthalten.\\
				Somit ist $\pi$ ein nicht-trivialer Automorphismus von $(\N, \mathbb{P}, \cdot)$.\\
			\item	Es gibt den folgenden nicht-trivialen Automorphismus: $\pi(1) = 1$, $\pi(2) = 4$, $\pi(4) = 2$ und $\pi(3) = 3$.\\
				$1$ und $3$ sind offensichtlich unterscheidbar ($1$ enthält nur ausgehende Kanten, $3$ nur eingehende Kanten). $2$ und $4$ sind jedoch nicht voneinander unterscheidbar. Es gilt hier nämlich $\varphi \coloneqq \forall x((E2x \leftrightarrow E4x) \land (Ex2 \leftrightarrow Ex4))$. Aus diesem Grund sind beide Knoten nicht voneinander unterscheidbar und wir können sie mit $\pi$ "vertauschen". Dann gilt offensichtlich aufgrund von $\varphi$ $\forall x ((E2x \leftrightarrow E\pi(2)\pi(x) \land (Ex2 \leftrightarrow E\pi(x)\pi(2)))$\\
				Somit ist $\pi$ ein nicht-trivialer Automorphismus von $(\N, \mathbb{P}, \cdot)$.\\
			\item	Angenommen es gäbe eine Permutation $\pi$ die ein nicht-trivialer Automorphismus ist. Aufgrund von $\pi(0) = 0$ muss $\pi$ $0$ auf $0$ abbilden.\\
					Ab jetzt können wir so argumentieren, wie in der Vorlesung argumentiert wurde über $(\mathbb{N}, <)$. Angenommen wir bilden $1$ nicht auf $1$ ab. Dann müssen wir $1$ auf eine größere Zahl abbilden, da sonst $\pi(0) \nless \pi(1)$. Hier stoßen wir auf das Problem, dass wir keine andere Zahl auf $1$ abbilden können. Würden wir eine negative Zahl $a$ auf $1$ abbilden, so wäre $\pi(a) \nless \pi(0)$. Würden wir eine positive Zahl $b$ auf $1$ abbilden (folglich $1 < b$), so wäre $\pi(1) \nless \pi(b)$. Deshalb muss auch $1$ auf $1$ abgebildet werden.\\
					Per Induktion sehen wir wie in der Vorlesung, dass alle positiven Zahlen auf sich selber abgebildet werden müssen. Äquivalent gehen wir für die negativen Zahlen vor.\\
					Somit ist die Identität der einzige Automorphismus auf $(\mathbb{Z}, 0, <)$.
			\end{enumerate}
	\item	Wir wissen unter anderem aus Hausaufgabe 05 Aufgabe 04, dass solche Strukturen ein Nullelement, das sogenannte leere Wort $\epsilon$ besitzen, welches wir mit einer einfachen Formel $\varphi$ bestimmen können. $\varphi(x)$ sei also genau dann erfüllt, wenn $x$ das leere Wort $\epsilon$ ist.\\
			Für jeden Automorphismus $\pi$ auf $\mathfrak{A}$ muss gelten, dass $\forall x \forall y \pi(x) \circ \pi(y) = \pi(x \circ y)$. Es muss also auch gelten: $\exists x \forall y (\varphi(x) \land \pi(x) \circ \pi(y) = \pi(x \circ y))$. Da $x$ also das leere Wort ist, ist $x \circ y = y$. $\pi(x) \circ \pi(y) = \pi(y)$ ist nur dann für alle $y$ erfüllt, wenn $\pi(x)$ das leere Wort ist. Also ist $\pi(\epsilon) = \epsilon$, nachdem wir $\epsilon$ als das leere Wort bestimmt haben.\\
			Eine weitere Eigenschaft von jedem $\pi$ ist, dass $\vert \pi(x) \vert = \vert x \vert$ für jedes $x$ sein muss, also $\pi$ die Längen der Elemente nicht verändern darf. \textit{Hab grad kp, bin aber sicher.}\\
			\textit{Denke sowas wie alle} $a \mapsto b, b \mapsto a$.
\end{enumerate}


\newpage


\section*{Aufgabe 6}
\begin{enumerate}[label=(\alph*)]
	\item	Ist nur möglich, wenn $1$ in $(\mathbb{Q},0,+,<)$ elementar definierbar ist.\\
			$1$ ist jedoch nicht elementar definierbar.\\
			\textit{Beweis?}
	\item	$$\varphi(n,p) \coloneqq (n < p \lor n=p) \land 1 < p \land (\forall y \forall z (y \cdot z = p \rightarrow (y=1 \lor z=1)))$$
			Besteht aus $n \leq p$ und $p \in \mathbb{P}$.
	\item	
	\item	$$\varphi(x) \coloneqq x \circ 0 = 1 \land D(x) = x$$
			Siehe \href{https://de.wikipedia.org/wiki/Exponentialfunktion#Ableitung}{\underline{Wikipedia - Exponentialfunktion - Ableitung}}.
\end{enumerate}


\end{document}
