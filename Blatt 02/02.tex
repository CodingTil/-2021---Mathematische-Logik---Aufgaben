\documentclass[a4paper,11pt]{article}

% Layout
\usepackage[a4paper, left=3cm, right=3cm, top=2cm, bottom=3cm]{geometry} % kleinere Ränder
\usepackage{parskip}

% Umlaute in der Datei erlauben, auf Deutsch umstellen
\usepackage[T1]{fontenc}
\usepackage{lmodern}
\usepackage[utf8]{inputenc}
\usepackage[english, ngerman]{babel}
%\usepackage[english]{babel} % for submissions in ENGLISH

% Mathesymbole und Ähnliches
\usepackage{amsmath}
\usepackage{mathtools}
\usepackage{amssymb}
\usepackage{microtype}
\usepackage{stmaryrd}
\usepackage{enumitem}

% Grafiken und PDFs einfügen
\usepackage{graphicx}
\usepackage{pdfpages}

% PDF-Tools
\usepackage[hidelinks, unicode]{hyperref}

% Abbildungen
\usepackage{tikz}
\usetikzlibrary{arrows,calc}

% Reelle, Natürliche, Ganze, Rationale Zahlen
\newcommand{\R}{\ensuremath{\mathbb{R}}}
\newcommand{\N}{\ensuremath{\mathbb{N}}}
\newcommand{\Z}{\ensuremath{\mathbb{Z}}}
\newcommand{\Q}{\ensuremath{\mathbb{Q}}}

% Fraktur für Strukturen
\newcommand{\A}{\ensuremath{\mathfrak A}}
\newcommand{\B}{\ensuremath{\mathfrak B}}
\newcommand{\I}{\ensuremath{\mathfrak I}}

% Makros für logische Operatoren
\newcommand{\xor}{\ensuremath{\oplus}} % exklusives oder
\newcommand{\impl}{\ensuremath{\rightarrow}} % logische Implikation

% Meistens ist \varphi schöner als \phi, genauso bei \theta
\renewcommand{\phi}{\varphi}
\renewcommand{\theta}{\vartheta}

% Aufzählungen anpassen (alternativ: \arabic, \alph)
\renewcommand{\labelenumi}{(\roman{enumi})}

% BITTE NICHT ÄNDERN: interne Kommandos für die Informationen (Blattnummer, Gruppe, ...)
% PLEASE DO NOT EDIT THIS SECTION

\newcommand{\printsheet}{?}
\newcommand{\sheet}[1]{%
\renewcommand{\printsheet}{#1}%
}

\newcommand{\printgroup}{?}
\newcommand{\group}[1]{%
\renewcommand{\printgroup}{#1}%
}

\newcommand{\printmembers}{}
\newcommand{\printmember}[3]{{#2} {#3} & {#1} \\}

\newcommand{\pdfmembers}{}
\newcommand{\pdfmember}[3]{{#1} {#3}, {#2}; }

\newcommand{\member}[3]{%
\expandafter\renewcommand\expandafter\printmembers\expandafter{\printmembers\printmember{#1}{#2}{#3}}%
\expandafter\renewcommand\expandafter\pdfmembers\expandafter{\pdfmembers\pdfmember{#1}{#2}{#3}}%
}

\AtBeginDocument{\hypersetup{
    pdftitle = {Übungsblatt \printsheet},
    pdfauthor = {Abgabegruppe \printgroup: \pdfmembers},
    pdfsubject = {Mathematische Logik}
}}

% <-------- HIER alle Informationen eintragen ========================
% enter your information here
\sheet{02} % Nummer des Blatts / number of exercise sheet
\group{55} % Gruppennummer der Abgabegruppe in Moodle / group number from Moodle
% Die Gruppennummer erscheint NICHT auf dem Blatt (nur in den PDF-Metadaten).
% The group number does NOT appear on the sheet (check the PDF meta data).

% alle Gruppenmitglieder in der Form \member{Matrikelnummer}{Vorname}{Nachname}
% group members are entered as \member{matriculation number}{first name}{last name}
\member{405401}{Marc}{Ludevid}
\member{405409}{Andrés}{Montoya}
\member{405959}{Til}{Mohr}
%\member{999999}{Viertes}{Mitglied}

\begin{document}

% Platz für die Punktetabelle und Kommentare
\hfill
\begin{Form}
\begin{tabular}{c}
\\
Gesamtpunkte: \\[2mm]
\TextField[name=points, width=20mm, align=1, bordercolor={0 0 0}]{} \\
\\
\end{tabular}
\end{Form}

% Kopfzeile
{\raggedright
\begin{tabular}{l}
    MaLo \\
    SS 2021 \\
    \today{} \\
\end{tabular}}
\hfill
{\Large Übungsblatt \printsheet}
\hfill
\begin{tabular}{l l}
\printmembers
\end{tabular}
\hrule


% <-------- HIER beginnt die Lösung ========================
% your SOLUTION starts here

\section*{Aufgabe 1}
E-Test

\section*{Aufgabe 2}
\begin{enumerate}[label=(\alph*)]
	\item 	Aus der VL wissen wir, dass $\{\neg, \land\}$ funktional vollständig ist.
			\begin{itemize}
				\item $\neg x \equiv m(x,x,x,x)$
				\item $x \land y \equiv m(x,x,y,m(x,x,x,x))$
			\end{itemize}
			Da $\{\neg, \land\}$ funktional vollständig ist und wir dies mit $\{m\}$ darstellen können, ist auch $\{m\}$ funktional vollständig.
			
	\item	Wir zeigen mittels Induktion, dass die Boolsche Funktion $\neg$ sich nicht aus $\{\impl, u, 1\}$ darstellen lässt.
			\begin{itemize}
				\item[I.A.]	Konstante $1$ \\
							Ausgangsvariable $x \in \tau$
				\item[I.S.]	$\varphi = \psi \impl \theta$, falls $\psi$ und $\theta$ bereits gebildete Formeln sind \\
							$\varphi = u(\psi, \theta, \lambda)$, falls $\psi$, $\theta$ und $\lambda$ bereits gebildete Formeln sind
			\end{itemize}
			Hieraus erkennt man jedoch, dass $\varphi$ nie $\neg$ darstellen kann, da: \\
			$1 \impl x \equiv x \impl x \equiv x \equiv u(1,1,x) \equiv u(1,x,1) \equiv u(x,1,1) \equiv u(x,x,x)$ \\
			und \\
			$x \impl 1 \equiv 1 \equiv u(1,x,x) \equiv u(x,1,x) \equiv u(x,x,1)$\\
			Aus diesem Grund ist $\{\impl, u, 1\}$ nicht funktional vollständig.

	\item	Für $f \in B^n$ beliebig gilt laut Aufgabenstellung, dass $f$ nicht monoton, also gibt es $a, b \in \{0,1\}^n$ mit $a \leq b$ für die gilt: $f(a) \not \leq f(b)$, also $f(a) > f(b)$ bzw. $f(a) = 1$ und $f(b) = 0$. Sei ein Paar $a,b$ gegeben die diese Bedingung erfüllen. Wenn es mehrere gibt wähle eins, sodass b minimal ist wenn dieses als Binärzahl interpretiert wird. Da $a$ und $b$ nicht gleich sind (andernfalls wäre $f(a) = f(b)$) gilt $a < b$ und somit gibt es einen Index $0 \leq i < n$ sodass $a_i < b_i$. Dementsprechen ist $a_i = 0$ und $b_i = 1$.

			Außerdem wissen wir, dass aufgrund der Minimalität von $b$, wenn man $b$ so zu $b_{i=0}$ abwandelt dass man an der Stelle mit Index $i$ eine $0$ einfügt, dann $f(b_{i=0}) = 1$. Beweis: Andernfalls wäre $a, b_{i=0}$ ebenfalls ein Paar, für das sowohl $a \leq b_{i=0}$, wegen $a_i = 0$, als auch $f(a) \not \leq f(b_{i=0})$ gilt, weil $f(b_{i=0}) = 0$ wäre. $b_{i=0}$ ist als Binärzahl kleiner als $b$ was der Voraussetzung widerspricht, dass $b$ minimal gewählt wurde.

			Schließlich definieren wir die Funktion $b': \{0,1\} \rightarrow \{0,1\}$ sodass $b'(X) = f(b_{i=X})$ wobei $b_{i=X}$ ein abgeändertes $b$ ist, wobei $b_i$ auf den Wert $X$ gesetzt wurde.

			Wir wissen also:

			\begin{itemize}
    			\item $b'(1) \equiv 0$, denn für jedes Paar $a,b$ für das die Bedingungen der nicht-Monotinie gelten, $b_i = 1$ sein muss. Somit ist $b'(1) = f(b)$ denn $b$ wird nicht abgeändert.
   				\item $b'(0) \equiv 1$, denn wenn $b'(0) \equiv 0$ gelten würde, $a, b_{i=0}$ ein Paar wäre das die Bedingung der nicht-Monotonie erfüllt und somit $b$ nicht minimal im Sinne einer binären Zahl wäre.
			\end{itemize}

			Es gilt also:

			$$
    			\neg X = b'(X)
			$$

			Somit haben wir die Negation aus $b'$ und implizit aus $f$ abgeleitet und somit ist die gegebene Menge funktional vollständig.
\end{enumerate}


\newpage


\section*{Aufgabe 3}
\begin{enumerate}[label=(\alph*)]
	\item	\begin{align*}
				M_0 &= \emptyset \\
				M_1 &= \{B\} \\
				M_2 &= \{B, D\} \\
				M_3 &= \{B, D, F, A\} \\
				M_4 &= \{B, D, F, A\} \coloneqq M
			\end{align*}
			Der Algorithmus terminiert.\\
			Das Minimale Modell ist: $\mathfrak{I}: A \mapsto 1, B \mapsto 1, C \mapsto 0, D \mapsto 1, E \mapsto 0, F \mapsto 1$
	\item	$\Phi$ ist offensichtlich äquivalent zu $\Phi' \coloneqq \{X \impl Y, X \land Z \impl Y\}$ und $\psi$ äquivalent zu $\psi' \coloneqq (X \impl X) \land (X \impl Y) \land (X \impl Z) = (X \impl Y) \land (X \impl Z)$\\
			$\Phi'$ besteht also nur aus Horn-Formeln und $\psi'$ selber ist eine Horn-Formel.\\
			Da $\Phi \models \psi$ bzw. $\Phi' \models \psi'$ genau dann gilt, wenn jedes Modell von $\Phi'$ auch ein Modell von $\psi'$ ist, und wir hier eben nur Horn-Formeln haben, gilt $\Phi \models \psi$ eben auch genau dann, wenn das minimale Modell von $\Phi$ dem von $\psi$ entspricht.\\
			Der Markierungsalgorithmus liefert uns für alle Horn-Formeln das minimale Modell $\mathfrak{I}: X,Y,Z \mapsto 0$. Also gilt $\Phi \models \psi$.
\end{enumerate}


\newpage


\section*{Aufgabe 4}
\begin{enumerate}[label=(\alph*)]
	\item	$\varphi = \bigwedge_{i=1}^{n} \varphi_i$ mit $\varphi_i = \begin{cases}(\bigwedge_{j=1}^{m_i-1} X_{i,j}) \impl X_{i,m} \\ \bigwedge_{j=1}^{m_i} X_{i,j} \end{cases}$.\\
			Offensichtlich gilt für ein $\mathfrak{I} \models \varphi$ auch $\mathfrak{I} \models \varphi_i$ für alle $i$ in $\varphi$.\\
			Sei nun $\mathfrak{I}_1 \models \varphi$, $\mathfrak{I}_2 \models \varphi$. Für jedes $i$ in $\varphi$ unterscheiden wir nun 2 Fälle:
			\begin{itemize}
				\item	Falls für alle $1 \leq j \leq m_i$ gilt: $\mathfrak{I}_1(X_{i,j}) = 1 = \mathfrak{I}_2(X_{i,j})$, dann ist auch $(\mathfrak{I}_1 \cap \mathfrak{I}_2)(X_{i,j}) = 1$, weshalb $\mathfrak{I}_1 \cap \mathfrak{I}_2 \models \varphi_i$ stimmt.
				\item	Falls für ein $1 \leq j \leq m_i$ gilt: $\mathfrak{I}_1(X_{i,j}) = 0$ oder $\mathfrak{I}_2(X_{i,j}) = 0$, dann ist auch $(\mathfrak{I}_1 \cap \mathfrak{I}_2)(X_{i,j}) = 0$, weshalb $\mathfrak{I}_1 \cap \mathfrak{I}_2 \models \varphi_i$ stimmt.
			\end{itemize}
			Also gilt $\mathfrak{I}_1 \cap \mathfrak{I}_2 \models \varphi_i$ für alle $i$ in $\varphi$, weshalb auch $\mathfrak{I}_1 \cap \mathfrak{I}_2 \models \varphi$ gelten muss.
	
	\item	Da Horn-Formeln unter Schnitt abgeschlossen sind, muss es auch immer ein eindeutiges kleinstes Modell zu einer Horn-Formel $\varphi$ geben:
			Gäbe es kein eindeutiges kleinstes Modell, sondern 2 voneinander verschiedene minimale Modelle $\mathfrak{I}_1, \mathfrak{I}_2$ so wäre $\mathfrak{I}_1 \cap \mathfrak{I}_2 \not\models \varphi$, da $\mathfrak{I}_1 \cap \mathfrak{I}_2 \leq \mathfrak{I}_1$ und $\mathfrak{I}_1 \cap \mathfrak{I}_2 \leq \mathfrak{I}_2$, jedoch $\mathfrak{I}_1, \mathfrak{I}_2$ minimal sind, also insbesondere $\mathfrak{I}_1 \cap \mathfrak{I}_2$ nicht minimal.\\
			Widerspruch!\\
			Es muss immer ein eindeutiges kleinstes Modell zu einer Horn-Formel $\varphi$ geben!
	
	\item	\begin{itemize}
				\item 	$\mathfrak{I}_1: T,R,U \mapsto 1; S \mapsto 0$ und $\mathfrak{I}_2: T,S,R \mapsto 1; U \mapsto 0$ sind Modelle von $\varphi_1$, jedoch ist $\mathfrak{I}_1 \cap \mathfrak{I}_2: T,R \mapsto 1; S,U \mapsto 0$ kein Modell von $\varphi_1$. Da jedoch Horn-Formeln unter Schnitt abgeschlossen sind, ist $\varphi_1$ nicht äquivalent zu einer Horn-Formel.
				\item	\begin{align*}
							\varphi_2 	&\equiv (\neg A \impl (B \lor C)) \land (\neg B \impl (A \lor C)) \land (\neg C \impl (A \lor B)) \\
										&\equiv (A \lor B \lor C) \land (A \lor B \lor C) \land (A \lor B \lor C) \\
										&\equiv A \lor B \lor C
						\end{align*}
						Auch dies ist offensichtlich keine Horn-Formel aus derselben Begründung. Zudem darf eine Klausel in einer Horn-Formel höchstens ein positives Literal vorkommen. Hier sind es aber 3.
			\end{itemize}
\end{enumerate}


\newpage


\section*{Aufgabe 5}
\begin{enumerate}[label=(\alph*)]
	\item	\begin{enumerate}[label=(\roman*)]
				\item	Diese Aussage ist richtig.\\
						Wie oben bereits gesagt, gilt $\Phi \models \psi$ offensichtlich für alle $\psi \in \Psi$, wenn $\Phi \models \bigwedge \Psi$ gilt. Also muss auch für jedes $\Psi_0 \subseteq \Psi$ $\Phi \models \psi_0$ für alle $\psi_0 \in \Psi_0$ gelten, also auch $\Phi \models \Psi_0$.
				\item	Diese Aussage ist falsch.\\
						Sei $\Phi = \Psi = \{X\}$ und $\Psi_0 = \{X, \neg X\}$. Es gilt zwar $\Phi \models \Psi$, jedoch nicht $\Phi \models \Psi_0$!
			\end{enumerate}
	
	\item	Angenommen $\Phi$ ist erfüllbar und es gilt $\Phi \models \psi$. Dann gibt es also ein Modell $\mathfrak{I}$ zu $\Phi$, welches auch Modell von $\psi$ ist. Dann kann aber $\Phi \models \neg \psi$ nicht gelten, da dieses Modell $\mathfrak{I}$ kein Modell von $\neg \psi$ sein kann. Dies führt zum Widerspruch. Also kann $\Phi$ nicht erfüllbar sein.
	
	\item	Da $\Psi_i \cap \Psi_{i+1} = \Psi_{i+1}$ für alle $i \in \mathbb{N}$, ist $\bigcap_{i\in\mathbb{N}} \Psi_i \models \vartheta$
\end{enumerate}





\section*{Aufgabe 6}
$\Phi \coloneqq \{X_u \xor X_v \vert \{u,v\} \in E\}$\\
Falls $\Phi$ erfüllbar ist, dann gibt es ein Modell $\mathfrak{I}$ für $\Phi$. Für alle $v \in V$ gilt dann:\\
Falls $\mathfrak{I}(X_v) = 0$, dann ist $v \in W_0$.\\
Falls $\mathfrak{I}(X_v) = 1$, dann ist $v \in W_1$.\\
$G$ ist genau dann bipartit, wenn $\Phi$ erfüllbar ist. Nach dem Kompaktheitssatz ist $\Phi$ genau dann erfüllbar, wenn jede endliche Teilmenge $\Phi_0$ von $\Phi$ erfüllbar ist, also jeder endliche Teilgraph von $G$ bipartit ist.\\
Folglich ist $G$ genau dann erfüllbar, wen jeder endliche Teilgraph von $G$ bipartit ist.

\end{document}
