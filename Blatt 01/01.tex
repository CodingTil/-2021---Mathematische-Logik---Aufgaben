\documentclass[a4paper,11pt]{article}

% Layout
\usepackage[a4paper, left=3cm, right=3cm, top=2cm, bottom=3cm]{geometry} % kleinere Ränder
\usepackage{parskip}

% Umlaute in der Datei erlauben, auf Deutsch umstellen
\usepackage[T1]{fontenc}
\usepackage{lmodern}
\usepackage[utf8]{inputenc}
\usepackage[english, ngerman]{babel}
%\usepackage[english]{babel} % for submissions in ENGLISH

% Mathesymbole und Ähnliches
\usepackage{amsmath}
\usepackage{mathtools}
\usepackage{amssymb}
\usepackage{microtype}
\usepackage{stmaryrd}

% Grafiken und PDFs einfügen
\usepackage{graphicx}
\usepackage{pdfpages}

% PDF-Tools
\usepackage[hidelinks, unicode]{hyperref}

% Abbildungen
\usepackage{tikz}
\usetikzlibrary{arrows,calc}

% Reelle, Natürliche, Ganze, Rationale Zahlen
\newcommand{\R}{\ensuremath{\mathbb{R}}}
\newcommand{\N}{\ensuremath{\mathbb{N}}}
\newcommand{\Z}{\ensuremath{\mathbb{Z}}}
\newcommand{\Q}{\ensuremath{\mathbb{Q}}}

% Fraktur für Strukturen
\newcommand{\A}{\ensuremath{\mathfrak A}}
\newcommand{\B}{\ensuremath{\mathfrak B}}
\newcommand{\I}{\ensuremath{\mathfrak I}}

% Makros für logische Operatoren
\newcommand{\xor}{\ensuremath{\oplus}} % exklusives oder
\newcommand{\impl}{\ensuremath{\rightarrow}} % logische Implikation

% Meistens ist \varphi schöner als \phi, genauso bei \theta
\renewcommand{\phi}{\varphi}
\renewcommand{\theta}{\vartheta}

% Aufzählungen anpassen (alternativ: \arabic, \alph)
\renewcommand{\labelenumi}{(\roman{enumi})}

% BITTE NICHT ÄNDERN: interne Kommandos für die Informationen (Blattnummer, Gruppe, ...)
% PLEASE DO NOT EDIT THIS SECTION

\newcommand{\printsheet}{?}
\newcommand{\sheet}[1]{%
\renewcommand{\printsheet}{#1}%
}

\newcommand{\printgroup}{?}
\newcommand{\group}[1]{%
\renewcommand{\printgroup}{#1}%
}

\newcommand{\printmembers}{}
\newcommand{\printmember}[3]{{#2} {#3} & {#1} \\}

\newcommand{\pdfmembers}{}
\newcommand{\pdfmember}[3]{{#1} {#3}, {#2}; }

\newcommand{\member}[3]{%
\expandafter\renewcommand\expandafter\printmembers\expandafter{\printmembers\printmember{#1}{#2}{#3}}%
\expandafter\renewcommand\expandafter\pdfmembers\expandafter{\pdfmembers\pdfmember{#1}{#2}{#3}}%
}

\AtBeginDocument{\hypersetup{
    pdftitle = {Übungsblatt \printsheet},
    pdfauthor = {Abgabegruppe \printgroup: \pdfmembers},
    pdfsubject = {Mathematische Logik}
}}

% <-------- HIER alle Informationen eintragen ========================
% enter your information here
\sheet{X} % Nummer des Blatts / number of exercise sheet
\group{55} % Gruppennummer der Abgabegruppe in Moodle / group number from Moodle
% Die Gruppennummer erscheint NICHT auf dem Blatt (nur in den PDF-Metadaten).
% The group number does NOT appear on the sheet (check the PDF meta data).

% alle Gruppenmitglieder in der Form \member{Matrikelnummer}{Vorname}{Nachname}
% group members are entered as \member{matriculation number}{first name}{last name}
\member{405401}{Marc}{Ludevid}
\member{405409}{Andrés}{Montoya}
\member{405959}{Til}{Mohr}
%\member{999999}{Viertes}{Mitglied}

\begin{document}

% Platz für die Punktetabelle und Kommentare
\hfill
\begin{Form}
\begin{tabular}{c}
\\
Gesamtpunkte: \\[2mm]
\TextField[name=points, width=20mm, align=1, bordercolor={0 0 0}]{} \\
\\
\end{tabular}
\end{Form}

% Kopfzeile
{\raggedright
\begin{tabular}{l}
    MaLo \\
    SS 2021 \\
    \today{} \\
\end{tabular}}
\hfill
{\Large Übungsblatt \printsheet}
\hfill
\begin{tabular}{l l}
\printmembers
\end{tabular}
\hrule


% <-------- HIER beginnt die Lösung ========================
% your SOLUTION starts here

\section*{Aufgabe 1}
E-Test

\section*{Aufgabe 2}

\renewcommand{\labelenumi}{(\alph{enumi})}
\begin{enumerate}
	\item	Zu Beginn weisen wir jedem der Module ``Algebra'', ``Computeralgebra'', ``Ethik'', ``Formale Systeme'' und ``Grundlagen der Mathematik'' eine Aussagenvariable zu. Wir bekommen also:
			$$ \tau \coloneqq \{ A,C,E,F,G \} $$
			Diese Variablen haben folgende Semantik. Definieren wir $J$, die aussagenlogische Interpretation. Dann gilt:
			\begin{itemize}
				\item $J(A)$ gdw. ``Algebra'' belegt wurde
				\item $J(C)$ gdw. ``Computeralgebra'' belegt wurde
				\item $J(E)$ gdw. ``Ethik'' belegt wurde
				\item $J(F)$ gdw. ``Formale Systeme'' belegt wurde
				\item $J(G)$ gdw. ``Grundlagen der Mathematik'' belegt wurde
			\end{itemize}


	\item	\begin{itemize}
				\item $\phi_{i} \coloneqq \neg (A \land C \land E \land F \land G) \land (A \lor C \lor E \lor F \lor G)$
				\item $\phi_{ii} \coloneqq C \impl (E \land F)$	
				\item $\phi_{iii} \coloneqq (A \impl G) \land (G \impl A)$
				\item $\phi_{iv} \coloneqq A \impl C$
				\item $\phi_{v} \coloneqq \neg F$
				\item $\phi_{vi} \coloneqq G$
				\item $\phi_{vii} \coloneqq ((\neg F) \lor G) \land \neg ((\neg F) \land G)$
			\end{itemize}


	\item	Aus (b) können wir nun $\phi$ definieren:
			$$ \phi \coloneqq \phi_{i} \land \phi_{ii} \land \phi_{iii} \land \phi_{iv} \land \phi_{v} \land \phi_{vi} \land \phi_{vii} \land $$
			
			\begin{itemize}
			\item[Fall 1]	
			\end{itemize}
\end{enumerate}


\section*{Aufgabe 3}
Einige offensichtliche Zwischenschritte werden wir hier zur Übersicht überspringen.
\renewcommand{\labelenumi}{(\alph{enumi})}
\begin{enumerate}
	\item
	\begin{itemize}
	\item[(i)]	\begin{itemize}
				\item[(1)]	$J(\phi) = 1$ gdw. $J((A \xor B) \xor (C \xor D)) = 1 = J((\neg (A \impl B) \land \neg (C \impl D))$
				\item[(2)]	$J((\neg (A \impl B) \land \neg (C \impl D)) = 1$ gdw. $J(A) = 0 = J(C)$ und $J(B) = 1 = J(D)$
				\item[(3)]	$J((A \xor B) \xor (C \xor D)) = 1$ gdw. genau eine Variable $X \in \{A,B,C,D\}$ $J(X) = 1$, alle anderen Variablen $Y \in {A,B,C,D}, Y \neq X$ $J(Y) = 0$
				\end{itemize}
				(2) und (3) stehen aber im Widerspruch. Daher ist $\phi$ unerfüllbar.
				
	\item[(ii)]	\begin{itemize}
				\item[(1)]	$J(\phi) = 1$ gdw. $J((X \impl \neg X) \lor (Y \impl \neg Y)) = 1 = J((X \impl 0) \xor (1 \impl Y))$
				\item[(2)]	$J((X \impl \neg X) \lor (Y \impl \neg Y)) = 1$ gdw. nicht sowohl $J(X) = 1$ und $J(Y) = 1$
				\item[(3)]	$J((X \impl 0) \xor (1 \impl Y)) = 1$ gdw. sowohl $J(X) = 1$ und $J(Y) = 1$ ODER $J(X) = 0$ und $J(Y) = 0$
				\end{itemize}
				$\phi$ ist also genau dann erfüllt, wenn $J(X) = 0$ und $J(Y) = 0$.\\
				\\Betrachte $\neg \phi$:
				$\neg \phi$ ist genau dann erfüllt, wenn $\phi$ nicht erfüllt ist. Folglich ist $J(\neg \phi) = 1$ gdw. $J(X) = 1$ und $J(Y) = 0$ ODER $J(X) = 1$ und $J(Y) = 1$ ODER $J(X) = 0$ und $J(Y) = 1$.\\
				\\Da sowohl $\phi$ als auch $\neg \phi$ erfüllbar sind, ist $\phi$ nicht-trivial.
	\end{itemize}
			
	\item
	\begin{equation}
	\begin{split}
	(\neg C \impl A) \land ((\neg A \impl C) \land \neg(\neg A \land \neg D)) &\Leftrightarrow (C \lor A) \land ((A \lor C) \land (A \lor D)) \\
		&\Leftrightarrow  (C \lor A) \land (A \lor (C \land D))) \\
		&\Leftrightarrow  A \lor (C \land (C \land D)) \\
		&\Leftrightarrow A \lor (C \land D) \\
		&\Leftrightarrow (A \land 1) \lor (C \land D) \\
		&\Leftrightarrow (A \land (0 \impl B)) \lor (C \land D) \\
		&\Leftrightarrow (A \land (1 \lor B)) \lor (C \land D) \\
		&\Leftrightarrow ((A \land 1) \lor (A \land B)) \lor (C \land D) \\
		&\Leftrightarrow (A \lor (A \land B)) \lor (C \land D)
	\end{split}
	\end{equation}
			 
\end{enumerate}



\section*{Aufgabe 4}
\renewcommand{\labelenumi}{(\alph{enumi})}
\begin{enumerate}
\item	$\phi \coloneqq \neg X_{11} \land X_{12} \land X_{13} \land \neg X_{21} \land X_{22} \land \neg X_{23} \land \neg X_{31} \land X_{32} \land \neg X_{33}$

\item	$\phi \coloneqq \bigvee\limits_{0 \leq i,j < n} (X_{ij} \land X_{ji})$\\
		Mit dieser Formel muss mindestens ein Knotenpaar $\{i,j\}$ beide Kanten $(i,j)$ und $(j,i)$ zwischen sich haben. Also gibt es immer mindestens einen Kreis der Länge 2 als induzierten Teilgraphen $H=(\{i,j\}, \{(i,j),(j,i)\})$.

\item	asdasd
\end{enumerate}

\section*{Aufgabe 5*}

\end{document}
