\documentclass[a4paper,11pt]{article}

% Layout
\usepackage[a4paper, left=3cm, right=3cm, top=2cm, bottom=3cm]{geometry} % kleinere Ränder
\usepackage{parskip}

% Umlaute in der Datei erlauben, auf Deutsch umstellen
\usepackage[T1]{fontenc}
\usepackage{lmodern}
\usepackage[utf8]{inputenc}
\usepackage[english, ngerman]{babel}
%\usepackage[english]{babel} % for submissions in ENGLISH

% Mathesymbole und Ähnliches
\usepackage{amsmath}
\usepackage{mathtools}
\usepackage{amssymb}
\usepackage{microtype}
\usepackage{stmaryrd}

% Grafiken und PDFs einfügen
\usepackage{graphicx}
\usepackage{pdfpages}

% PDF-Tools
\usepackage[hidelinks, unicode]{hyperref}

% Abbildungen
\usepackage{tikz}
\usetikzlibrary{arrows,calc}

% Reelle, Natürliche, Ganze, Rationale Zahlen
\newcommand{\R}{\ensuremath{\mathbb{R}}}
\newcommand{\N}{\ensuremath{\mathbb{N}}}
\newcommand{\Z}{\ensuremath{\mathbb{Z}}}
\newcommand{\Q}{\ensuremath{\mathbb{Q}}}

% Fraktur für Strukturen
\newcommand{\A}{\ensuremath{\mathfrak A}}
\newcommand{\B}{\ensuremath{\mathfrak B}}
\newcommand{\I}{\ensuremath{\mathfrak I}}

% Makros für logische Operatoren
\newcommand{\xor}{\ensuremath{\oplus}} % exklusives oder
\newcommand{\impl}{\ensuremath{\rightarrow}} % logische Implikation

% Meistens ist \varphi schöner als \phi, genauso bei \theta
\renewcommand{\phi}{\varphi}
\renewcommand{\theta}{\vartheta}

% Aufzählungen anpassen (alternativ: \arabic, \alph)
\renewcommand{\labelenumi}{(\roman{enumi})}

% BITTE NICHT ÄNDERN: interne Kommandos für die Informationen (Blattnummer, Gruppe, ...)
% PLEASE DO NOT EDIT THIS SECTION

\newcommand{\printsheet}{?}
\newcommand{\sheet}[1]{%
\renewcommand{\printsheet}{#1}%
}

\newcommand{\printgroup}{?}
\newcommand{\group}[1]{%
\renewcommand{\printgroup}{#1}%
}

\newcommand{\printmembers}{}
\newcommand{\printmember}[3]{{#2} {#3} & {#1} \\}

\newcommand{\pdfmembers}{}
\newcommand{\pdfmember}[3]{{#1} {#3}, {#2}; }

\newcommand{\member}[3]{%
\expandafter\renewcommand\expandafter\printmembers\expandafter{\printmembers\printmember{#1}{#2}{#3}}%
\expandafter\renewcommand\expandafter\pdfmembers\expandafter{\pdfmembers\pdfmember{#1}{#2}{#3}}%
}

\AtBeginDocument{\hypersetup{
    pdftitle = {Übungsblatt \printsheet},
    pdfauthor = {Abgabegruppe \printgroup: \pdfmembers},
    pdfsubject = {Mathematische Logik}
}}

% <-------- HIER alle Informationen eintragen ========================
% enter your information here
\sheet{X} % Nummer des Blatts / number of exercise sheet
\group{55} % Gruppennummer der Abgabegruppe in Moodle / group number from Moodle
% Die Gruppennummer erscheint NICHT auf dem Blatt (nur in den PDF-Metadaten).
% The group number does NOT appear on the sheet (check the PDF meta data).

% alle Gruppenmitglieder in der Form \member{Matrikelnummer}{Vorname}{Nachname}
% group members are entered as \member{matriculation number}{first name}{last name}
\member{405401}{Marc}{Ludevid}
\member{405409}{Andrés}{Montoya}
\member{405959}{Til}{Mohr}
%\member{999999}{Viertes}{Mitglied}

\begin{document}

% Platz für die Punktetabelle und Kommentare
\hfill
\begin{Form}
\begin{tabular}{c}
\\
Gesamtpunkte: \\[2mm]
\TextField[name=points, width=20mm, align=1, bordercolor={0 0 0}]{} \\
\\
\end{tabular}
\end{Form}

% Kopfzeile
{\raggedright
\begin{tabular}{l}
    MaLo \\
    SS 2021 \\
    \today{} \\
\end{tabular}}
\hfill
{\Large Übungsblatt \printsheet}
\hfill
\begin{tabular}{l l}
\printmembers
\end{tabular}
\hrule


% <-------- HIER beginnt die Lösung ========================
% your SOLUTION starts here

\section*{Aufgabe 1}
E-Test

\section*{Aufgabe 2}

\paragraph* { (a) }

Zu zeigen ist, dass man mit $m$ sowohl $\neg$ als auch $\land$ dargestellt werden kann, da bekannt ist, dass diese beiden Operatoren zusammen funktional vollständig sind.
\begin{itemize}
	\item $\neg X \equiv m(X, 0, 0, X)$
	\item $X \land Y \equiv m(1, X, Y, 0)$
\end{itemize}

\paragraph* { (b) }

Damit eine Menge von Boolschen Funktionen funktional vollständig ist muss es auch den Wert $0$ produzieren können (Negation einer Tautologie, also Unerfüllbarkeit). Wir zeigen nun, dass dies mit der gegebenen nicht Menge nicht möglich ist. Wir zeigen also, dass jede aus dieser Menge von Operatoren und einer beliebigen Menge von Variablen konstruierten AL erfüllbar ist.

Induktionsanfang:
\begin{itemize}
	\item $1$ ist offensichtlich erfüllbar.
	\item $X$ ist ebenfalls offensichtlich erfüllbar.
\end{itemize}

Induktionsschritt:

$\rightarrow$: Seien erfüllbare AL-Formeln $\varphi$ und $\psi$ gegeben. Dann ist $\varphi \rightarrow \psi$ ebenfalls erfüllbar, da jede Interpretation die $\psi$ erfüllt auch $\varphi \rightarrow \psi$ erfüllt, weil sowohl $0 \rightarrow 1$ als auch $1 \rightarrow 1$ equivalent zu $1$ sind.

Damit ist bewiesen, dass der Wert $0$ nicht produziert werden kann und somit ist die gegebene Menge nicht funktional vollständig.

\paragraph* { (c) }
Für $f \in B^n$ beliebig gilt laut Aufgabenstellung, dass $f$ nicht monoton, also gibt es $a, b \in \{0,1\}^n$ mit $a \leq b$ für die gilt: $f(a) \not \leq f(b)$, also $f(a) > f(b)$ bzw. $f(a) = 1$ und $f(b) = 0$. Sei ein Paar $a,b$ gegeben die diese Bedingung erfüllen. Wenn es mehrere gibt wähle eins, sodass b minimal ist wenn dieses als Binärzahl interpretiert wird. Da $a$ und $b$ nicht gleich sind (andernfalls wäre $f(a) = f(b)$) gilt $a < b$ und somit gibt es einen Index $0 \leq i < n$ sodass $a_i < b_i$. Dementsprechen ist $a_i = 0$ und $b_i = 1$.

Ausserdem wissen wir, dass aufgrund der Minimalität von $b$, wenn man $b$ so zu $b_{i=0}$ abwandelt dass man an der Stelle mit Index $i$ eine $0$ einfügt, dann $f(b_{i=0}) = 1$. Beweis: Andernfalls wäre $a, b_{i=0}$ ebenfalls ein Paar, für das sowohl $a \leq b_{i=0}$, wegen $a_i = 0$, als auch $f(a) \not \leq f(b_{i=0})$ gilt, weil $f(b_{i=0}) = 0$ wäre. $b_{i=0}$ ist als Binärzahl kleiner als $b$ was der Voraussetzung widerspricht, dass $b$ minimal gewählt wurde.

Schliesslich definieren wir die Funktion $b': \{0,1\} \rightarrow \{0,1\}$ sodass $b'(X) = f(b_{i=X})$ wobei $b_{i=X}$ ein abgeändertes $b$ ist, wobei $b_i$ auf den Wert $X$ gesetzt wurde.

Wir wissen also:

\begin{itemize}
	\item $b'(1) \equiv 0$, denn für jedes Paar $a,b$ für das die Bedingungen der nicht-Monotinie gelten, $b_i = 1$ sein muss. Somit ist $b'(1) = f(b)$ denn $b$ wird nicht abgeändert.
	\item $b'(0) \equiv 1$, denn wenn $b'(0) \equiv 0$ gelten würde, $a, b_{i=0}$ ein Paar wäre das die Bedingung der nicht-Monotonie erfüllt und somit $b$ nicht minimal im Sinne einer binären Zahl wäre.
\end{itemize}

Es gilt also:

\begin{equation}
	\neg X = b'(X)
\end{equation}

Somit haben wir die Negation aus $b'$ und implizit aus $f$ abgeleitet und somit ist die gegebene Menge funktional vollständig.

\end{document}
