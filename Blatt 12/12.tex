\documentclass[a4paper,11pt]{article}

% Layout
\usepackage[a4paper, left=3cm, right=3cm, top=2cm, bottom=3cm]{geometry} % kleinere Ränder
\usepackage{parskip}
\usepackage{rotating}

% Umlaute in der Datei erlauben, auf Deutsch umstellen
\usepackage[T1]{fontenc}
\usepackage{lmodern}
\usepackage[utf8]{inputenc}
\usepackage[english, ngerman]{babel}
%\usepackage[english]{babel} % for submissions in ENGLISH

% Mathesymbole und Ähnliches
\usepackage{amsmath}
\usepackage{mathtools}
\usepackage{amssymb}
\usepackage{microtype}
\usepackage{stmaryrd}
\usepackage{enumitem}
\usepackage{bussproofs}
\usepackage{bbm}

% Grafiken und PDFs einfügen
\usepackage{graphicx}
\usepackage{pdfpages}

% PDF-Tools
\usepackage[hidelinks, unicode]{hyperref}

%%%%%%%%%%%%%%%%%%%%%%%%%%%%%%%%%%%%%%%%
% TIKZ-EINSTELLUNGEN FÜR SPIELE
%%%%%%%%%%%%%%%%%%%%%%%%%%%%%%%%%%%%%%%%

% Abbildungen
\usepackage{tikz}

% einige Bibliotheken
\usetikzlibrary{calc, tikzmark}
\usetikzlibrary{positioning}
\usetikzlibrary{arrows, arrows.meta}
\usetikzlibrary{decorations.markings}

% macht Linien und Pfeile dicker
\tikzset{every picture/.style={thick, >={Latex[round, length=2.5mm, width=2.5mm]}}}
\tikzset{every path/.style={shorten <= 2pt, shorten >= 2pt}}

% Styles für die Knoten der Verifiziererin und des Falsifizierers
\tikzstyle{MCverifier} = [draw, rectangle, rounded corners=1em, semithick, minimum height=2em]
\tikzstyle{MCfalsifier} = [draw, rectangle, thick, minimum height=2em]

%%%%%%%%%%%%%%%%%%%%%%%%%%%%%%%%%%%%%%%%

% Reelle, Natürliche, Ganze, Rationale Zahlen
\newcommand{\R}{\ensuremath{\mathbb{R}}}
\newcommand{\N}{\ensuremath{\mathbb{N}}}
\newcommand{\Z}{\ensuremath{\mathbb{Z}}}
\newcommand{\Q}{\ensuremath{\mathbb{Q}}}

% Fraktur für Strukturen
\newcommand{\A}{\ensuremath{\mathfrak A}}
\newcommand{\B}{\ensuremath{\mathfrak B}}
\newcommand{\I}{\ensuremath{\mathfrak I}}

% Makros für logische Operatoren
\newcommand{\xor}{\ensuremath{\oplus}} % exklusives oder
\newcommand{\impl}{\ensuremath{\rightarrow}} % logische Implikation

% Meistens ist \varphi schöner als \phi, genauso bei \theta
\renewcommand{\phi}{\varphi}
\renewcommand{\theta}{\vartheta}

% Aufzählungen anpassen (alternativ: \arabic, \alph)
\renewcommand{\labelenumi}{(\roman{enumi})}

% BITTE NICHT ÄNDERN: interne Kommandos für die Informationen (Blattnummer, Gruppe, ...)
% PLEASE DO NOT EDIT THIS SECTION

\newcommand{\printsheet}{?}
\newcommand{\sheet}[1]{%
\renewcommand{\printsheet}{#1}%
}

\newcommand{\printgroup}{?}
\newcommand{\group}[1]{%
\renewcommand{\printgroup}{#1}%
}

\newcommand{\printmembers}{}
\newcommand{\printmember}[3]{{#2} {#3} & {#1} \\}

\newcommand{\pdfmembers}{}
\newcommand{\pdfmember}[3]{{#1} {#3}, {#2}; }

\newcommand{\member}[3]{%
\expandafter\renewcommand\expandafter\printmembers\expandafter{\printmembers\printmember{#1}{#2}{#3}}%
\expandafter\renewcommand\expandafter\pdfmembers\expandafter{\pdfmembers\pdfmember{#1}{#2}{#3}}%
}

\AtBeginDocument{\hypersetup{
    pdftitle = {Übungsblatt \printsheet},
    pdfauthor = {Abgabegruppe \printgroup: \pdfmembers},
    pdfsubject = {Mathematische Logik}
}}

% <-------- HIER alle Informationen eintragen ========================
% enter your information here
\sheet{12} % Nummer des Blatts / number of exercise sheet
\group{55} % Gruppennummer der Abgabegruppe in Moodle / group number from Moodle
% Die Gruppennummer erscheint NICHT auf dem Blatt (nur in den PDF-Metadaten).
% The group number does NOT appear on the sheet (check the PDF meta data).

% alle Gruppenmitglieder in der Form \member{Matrikelnummer}{Vorname}{Nachname}
% group members are entered as \member{matriculation number}{first name}{last name}
\member{405401}{Marc}{Ludevid}
\member{405409}{Andrés}{Montoya}
\member{405959}{Til}{Mohr}
%\member{999999}{Viertes}{Mitglied}

\begin{document}

% Platz für die Punktetabelle und Kommentare
\hfill
\begin{Form}
\begin{tabular}{c}
\\
Gesamtpunkte: \\[2mm]
\TextField[name=points, width=20mm, align=1, bordercolor={0 0 0}]{} \\
\\
\end{tabular}
\end{Form}

% Kopfzeile
{\raggedright
\begin{tabular}{l}
    MaLo \\
    SS 2021 \\
    \today{} \\
\end{tabular}}
\hfill
{\Large Übungsblatt \printsheet}
\hfill
\begin{tabular}{l l}
\printmembers
\end{tabular}
\hrule


% <-------- HIER beginnt die Lösung ========================
% your SOLUTION starts here

\section*{Aufgabe 1}
E-Test

\section*{Aufgabe 2}
\begin{enumerate}[label=(\alph*)]
	\item	Ist nicht definierbar, da wir mit $\Box$ und $\Diamond$ nur Nachfolger von $v$ bestimmen können, aber keine Vorgänger. Deshalb gibt es keine Formel hierfür.
	\item	$$\Diamond (Q \land \Box (\neg P))$$
	\item	Allgemeine Verschiedenheit bzw. Verschiedenheit ist nicht definierbar. Zudem, wenn es hierfür eine erfüllbare Formel gäbe, dann gäbe es auch ein Baummodell, welches die Formel erfüllt. Dort gibt es aber keine Knoten mit zwei Vorgängern.\\
			Widerspruch! Also ist (c) nicht definierbar.
	\item	Ähnlich zu (c): Gäbe es solch eine erfüllbare Formel, dann auch ein Baummodell, welches die Formel erfüllt. In Bäumen gibt es aber keine Selbstkanten. Widerspruch!
	\item	(e) ist nicht definierbar, da schon die vereinfachte Eigenschaft (Es gibt einen Weg von $v$ zu einem seiner Vorgänger aus, der nicht auf einem Kreis liegt.) nicht definierbar ist: Angenommen es gibt einen Weg von $v$ zu einem seiner Vorgänger. Dann liegt dieser automatisch auch auf einem Kreis, da der Vorgänger von $v$ einen Transition zu $v$ besitzt.
\end{enumerate}


\newpage


\section*{Aufgabe 3}
\begin{center}\begin{tabular}{c || c | c | c | c | c}
$\psi_{VW}$ & 5 & 6 & 7 & 8 & 9 \\
\hline\hline
0 & $\langle a \rangle 1$	& $\langle b \rangle 1$	& $\langle b \rangle 1$	& $\sim$	& $\sim$	\\
\hline
1 & $\langle a \rangle 1$	& $\langle b \rangle (\langle b \rangle 1)$	& $\langle b \rangle (\langle a \rangle 1)$	& $\langle b \rangle 1$	& $\langle b \rangle 1$	\\
\hline
2 & $\langle a \rangle 1$	& $\langle b \rangle (\langle b \rangle 1)$	& $\langle b \rangle (\langle a \rangle 1)$	& $\langle b \rangle 1$	& $\langle b \rangle 1$	\\
\hline
3 & $\langle a \rangle 1$	& $\langle b \rangle (\langle b \rangle 1)$	& $\langle b \rangle (\langle a \rangle 1)$	& $\langle b \rangle 1$	& $\langle b \rangle 1$	\\
\hline
4 & $\langle a \rangle 1$	& $\langle b \rangle (\langle b \rangle 1)$	& $\langle b \rangle (\langle a \rangle 1)$	& $\langle b \rangle 1$	& $\langle b \rangle 1$	

\end{tabular}\end{center}
$$Z = \{(0,8), (0,9)\}$$


\newpage


\section*{Aufgabe 4}
\begin{enumerate}[label=(\alph*)]
	\item	\begin{tikzpicture}
    			\node[shape=circle,draw=black] (0) at (0,0) {0};
   				\node[shape=circle,draw=black] (1) at (3,0) {1};
    			\node (Q) at (3.5,-0.5) {Q};

   				\path [->] (0) edge (1);
    			\path [->] (1) edge [loop above] (1);
			\end{tikzpicture}
			
			Offensichtlich erfüllt dieses Modell $\psi$. Es gibt kein Modell mit weniger Zuständen, denn es müsste dann ein Zustand existieren, der gleichzeitig in $Q$ und nicht in $Q$ ist. Widerspruch!
	\item	\begin{tikzpicture}
    			\node[shape=circle,draw=black] (0) at (0,0) {0};
   				\node[shape=circle,draw=black] (1) at (3,0) {1};
    			\node (Q1) at (3.5,-0.5) {Q};
   				\node[shape=circle,draw=black] (2) at (6,0) {2};
   				\node[shape=circle,draw=black] (3) at (9,0) {3};
   				\node[shape=circle,draw=black] (4) at (12,0) {4};
    			\node (Q4) at (12.5,-0.5) {Q};

   				\path [->] (0) edge (1);
    			\path [->] (1) edge (2);
    			\path [->] (2) edge (3);
    			\path [->] (3) edge (4);
			\end{tikzpicture}
	\item	$$\phi \coloneqq Q \land \Box (\neg Q \land \Box (\neg Q \land \Box (Q)))$$
\end{enumerate}


\newpage


\section*{Aufgabe 5}
\begin{enumerate}[label=(\alph*)]
	\item	$$\psi^*(x) \coloneqq \forall y (Exy \impl (Py \impl \exists x (Eyx \land Qx)))$$
	\item	$$\theta^*(x) \coloneqq \exists y (Eyx)$$
			$\theta^*(x)$ ist genau dann Wahr, wenn $x$ einen Vorgänger ($y$) hat. In der Modallogik können wir aber nur immer auf Nachfolger zugreifen, nicht auf Vorgänger. 
	\item	Sei $I = \{1, \dots, 42\}$.
			$$\phi \coloneqq P_1 \land \bigwedge_{i \in I\setminus\{1\}} \Diamond (P_i \land \bigwedge_{j \in I\setminus\{1,i\}} \neg P_j)$$
\end{enumerate}


\end{document}
