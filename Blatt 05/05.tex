\documentclass[a4paper,11pt]{article}

% Layout
\usepackage[a4paper, left=3cm, right=3cm, top=2cm, bottom=3cm]{geometry} % kleinere Ränder
\usepackage{parskip}

% Umlaute in der Datei erlauben, auf Deutsch umstellen
\usepackage[T1]{fontenc}
\usepackage{lmodern}
\usepackage[utf8]{inputenc}
\usepackage[english, ngerman]{babel}
%\usepackage[english]{babel} % for submissions in ENGLISH

% Mathesymbole und Ähnliches
\usepackage{amsmath}
\usepackage{mathtools}
\usepackage{amssymb}
\usepackage{microtype}
\usepackage{stmaryrd}
\usepackage{enumitem}
\usepackage{bussproofs}

% Grafiken und PDFs einfügen
\usepackage{graphicx}
\usepackage{pdfpages}

% PDF-Tools
\usepackage[hidelinks, unicode]{hyperref}

% Abbildungen
\usepackage{tikz}
\usetikzlibrary{arrows,calc}

% Reelle, Natürliche, Ganze, Rationale Zahlen
\newcommand{\R}{\ensuremath{\mathbb{R}}}
\newcommand{\N}{\ensuremath{\mathbb{N}}}
\newcommand{\Z}{\ensuremath{\mathbb{Z}}}
\newcommand{\Q}{\ensuremath{\mathbb{Q}}}

% Fraktur für Strukturen
\newcommand{\A}{\ensuremath{\mathfrak A}}
\newcommand{\B}{\ensuremath{\mathfrak B}}
\newcommand{\I}{\ensuremath{\mathfrak I}}

% Makros für logische Operatoren
\newcommand{\xor}{\ensuremath{\oplus}} % exklusives oder
\newcommand{\impl}{\ensuremath{\rightarrow}} % logische Implikation

% Meistens ist \varphi schöner als \phi, genauso bei \theta
\renewcommand{\phi}{\varphi}
\renewcommand{\theta}{\vartheta}

% Aufzählungen anpassen (alternativ: \arabic, \alph)
\renewcommand{\labelenumi}{(\roman{enumi})}

% BITTE NICHT ÄNDERN: interne Kommandos für die Informationen (Blattnummer, Gruppe, ...)
% PLEASE DO NOT EDIT THIS SECTION

\newcommand{\printsheet}{?}
\newcommand{\sheet}[1]{%
\renewcommand{\printsheet}{#1}%
}

\newcommand{\printgroup}{?}
\newcommand{\group}[1]{%
\renewcommand{\printgroup}{#1}%
}

\newcommand{\printmembers}{}
\newcommand{\printmember}[3]{{#2} {#3} & {#1} \\}

\newcommand{\pdfmembers}{}
\newcommand{\pdfmember}[3]{{#1} {#3}, {#2}; }

\newcommand{\member}[3]{%
\expandafter\renewcommand\expandafter\printmembers\expandafter{\printmembers\printmember{#1}{#2}{#3}}%
\expandafter\renewcommand\expandafter\pdfmembers\expandafter{\pdfmembers\pdfmember{#1}{#2}{#3}}%
}

\AtBeginDocument{\hypersetup{
    pdftitle = {Übungsblatt \printsheet},
    pdfauthor = {Abgabegruppe \printgroup: \pdfmembers},
    pdfsubject = {Mathematische Logik}
}}

% <-------- HIER alle Informationen eintragen ========================
% enter your information here
\sheet{05} % Nummer des Blatts / number of exercise sheet
\group{55} % Gruppennummer der Abgabegruppe in Moodle / group number from Moodle
% Die Gruppennummer erscheint NICHT auf dem Blatt (nur in den PDF-Metadaten).
% The group number does NOT appear on the sheet (check the PDF meta data).

% alle Gruppenmitglieder in der Form \member{Matrikelnummer}{Vorname}{Nachname}
% group members are entered as \member{matriculation number}{first name}{last name}
\member{405401}{Marc}{Ludevid}
\member{405409}{Andrés}{Montoya}
\member{405959}{Til}{Mohr}
%\member{999999}{Viertes}{Mitglied}

\begin{document}

% Platz für die Punktetabelle und Kommentare
\hfill
\begin{Form}
\begin{tabular}{c}
\\
Gesamtpunkte: \\[2mm]
\TextField[name=points, width=20mm, align=1, bordercolor={0 0 0}]{} \\
\\
\end{tabular}
\end{Form}

% Kopfzeile
{\raggedright
\begin{tabular}{l}
    MaLo \\
    SS 2021 \\
    \today{} \\
\end{tabular}}
\hfill
{\Large Übungsblatt \printsheet}
\hfill
\begin{tabular}{l l}
\printmembers
\end{tabular}
\hrule


% <-------- HIER beginnt die Lösung ========================
% your SOLUTION starts here

\section*{Aufgabe 1}
E-Test

\section*{Aufgabe 2}
\begin{enumerate}[label=(\alph*)]
	\item	\begin{enumerate}[label=(\roman*)]
				\item	Damit $\mathfrak{B} \coloneqq (\mathbb{N}, +, -, \cdot) \subseteq \mathfrak{R}$ gilt, muss $\mathbb{N} \subseteq \mathbb{R}$ und alle Funktionssymbole $+,-,\cdot$ aus $\mathfrak{B}$ eine Restriktion auf $\mathbb{N}$ und abgeschlossen sein. Für $0,1 \in \mathbb{N}$ ist jedoch  $0 - 1 \not\in \mathbb{N}$. Daher ist $\mathfrak{B}$ nicht $\{-\}$-abgeschlossen. Also: $\mathfrak{B} \nsubseteq \mathfrak{R}$.\\
			
						Die kleinste Substruktur, dessen Universum $\mathbb{N}$ enthält, ist $\mathfrak{B}' \coloneqq (\mathbb{Z}, +, -, \cdot)$:\\
						Es gilt offensichtlich $\mathbb{N} \subseteq \mathbb{Z} \subseteq \mathbb{R}$ und für alle $a,b \in \mathbb{Z}$ gilt nun auch $a+b, a-b, a \cdot b \in \mathbb{Z}$. Also sind die hier vorkommenden Funktionen $+,-,\cdot$ alle Restriktionen auf $\mathbb{Z}$ und abgeschlossen. Damit ist $\mathfrak{B}'$ eine Substruktur von $\mathfrak{R}$.\\
						$\mathfrak{B}'$ ist auch die kleinste Substruktur, dessen Universum $\mathbb{N}$ enthält : Würde man ein Element $c,0 \in \mathbb{Z}, c \not\in \mathbb{N}$, also folglich $0-c \in \mathbb{N}$, aus dem Universum entfernen, so wäre $0 - (0-c)$ nicht im Universum, weshalb man auch $0-c$ entfernen müsste, da die Struktur sonst nicht $\{-\}$-abgeschlossen ist. Dann würde das Universum dieser Struktur jedoch nicht mehr $\mathbb{N}$ enthalten!
				
				\item	$\mathfrak{B} \coloneqq (2\mathbb{Z}, +, - , \cdot) \subseteq \mathfrak{R}$
						\begin{itemize}
							\item 	$2\mathbb{Z} \subseteq \mathbb{R}$ ist offensichtlich
							\item	Für alle $a,b \in 2\mathbb{Z}$ gilt offensichtlich $a+b, a-b, a \cdot b \in 2\mathbb{Z}$. Damit sind $+,-,\cdot$ Restriktionen auf $2\mathbb{Z}$ und $\mathfrak{B}$ ist $\{+,-,\cdot\}$-abgeschlossen.
						\end{itemize}
			\end{enumerate}
						
	\item	\begin{enumerate}[label=(\roman*)]
				\item	$\mathfrak{B} \coloneqq (\{1\}, +, - , \cdot, {}^{-1}) \nsubseteq \mathfrak{Q}$\\
						$\mathfrak{B}$ ist nicht $\{+\}$-abgeschlossen, denn für $1 \in \{1\}$ ist $1+1 \not\in \{1\}$.\\
						
						Die kleinste Substruktur, dessen Universum $\{1\}$ enthält, ist $\mathfrak{B}' \coloneqq \mathfrak{Q}$:\\
						$\mathfrak{B}'$ ist offensichtlich aufgrund Gleichheit eine Substruktur zu sich ($\mathfrak{Q}$) selber.\\
						Damit jede Substruktur, die $\{1\}$ enthält, $\{+,-\}$-abgeschlossen ist, muss jede solche Substruktur offensichtlich $\mathbb{Z}$ enthalten. Soll eine solche Substruktur nun auch $\{{}^{-1}\}$-abgeschlossen sein, so muss sie $\{z^{-1} \mid z \in \mathbb{Z}\} = \{q \mid q \in \mathbb{Q}, 0 \leq \vert q \vert \leq 1\}$ enthalten. Da eine solche Substruktur zusätzlich $\{\cdot\}$-abgeschlossen sein soll, müssen nun auch alle Vielfachen davon vorkommen.\\
						Damit muss jede Substruktur, die $\{1\}$ enthalten soll, mindestens $\mathbb{Q}$ enthalten.
				
				\item	$\mathfrak{B} \coloneqq (\{0\}, +, - , \cdot, {}^{-1}) \subseteq \mathfrak{Q}$
						\begin{itemize}
							\item	Da $0 \in \mathbb{Q}$ ist $\{0\} \subseteq \mathbb{Q}$
							\item	Für $0 \in \{0\}$ gilt offensichtlich $0+0 = 0-0 = 0 \cdot 0 = 0^{-1} = 0 \in \{0\}$. Damit sind $+,-,\cdot,{}^{-1}$ Restriktionen auf $\{0\}$ und $\mathfrak{B}$ ist $\{+,-,\cdot,{}^{-1}\}$-abgeschlossen.
						\end{itemize}
			\end{enumerate}
			
	\item	\begin{enumerate}[label=(\roman*),ref=(\roman*)]
				\item	$\mathfrak{C} \coloneqq (B, \cup, \cap, \overline{\phantom{A}}) \nsubseteq \mathfrak{B}$\\
						$\mathfrak{C}$ ist nicht $\{\overline{\phantom{A}}\}$-abgeschlossen, da $\emptyset \in B$, aber $\overline{\emptyset} = \mathbb{N} \not\in A$.\\
						
						Da jede Substruktur von $\mathfrak{B}$ $\{\cup\}$-abgeschlossen sein soll und $B$ alle einelementigen Teilmengen von $\mathbb{N}$ enthält, muss in jeder Substruktur von $\mathfrak{B}$ $\{A \subseteq \mathbb{N}\}$ enthalten sein, da man mit jeder einelementigen Teilmenge jede Teilmenge von $\mathbb{N}$ konstruieren kann.\\
						Folglich ist die kleinste Substruktur, die $B$ enthält, $\mathfrak{B}$ selber.
						\label{marker}
				
				\item	$\mathfrak{C} \coloneqq (B, \cup, \cap, \overline{\phantom{A}}) \nsubseteq \mathfrak{B}$\\
						$\mathfrak{C}$ ist nicht $\{\cap\}$-abgeschlossen, da $2\mathbb{N}, 2\mathbb{N}+1 \in B$, aber $2\mathbb{N} \cap 2\mathbb{N}+1 = \emptyset \not\in B$.\\
						
						Da jede Substruktur von $\mathfrak{B}$ $\{\cap\}$-abgeschlossen sein soll und $B$ für jedes gerade $n \in \mathbb{N}$ die Menge ${2\mathbb{N} + 1}_{n} \coloneqq 2\mathbb{N} \cup \{n\}$ bzw. für jedes ungerade $m \in \mathbb{N}$ die Menge ${2\mathbb{N}}_{m} \coloneqq 2\mathbb{N} \cup \{m\}$ besitzt, muss jede Substruktur die einelementigen Mengen $\{n\} = {2\mathbb{N} + 1}_{n} \cap 2\mathbb{N}+1$ bzw. $\{m\} = {2\mathbb{N}}_{m} \cap 2\mathbb{N}$ besitzen.\\
						Rest: analog zu \ref{marker}\\
						Folglich ist die kleinste Substruktur, die $B$ enthält, $\mathfrak{B}$ selber.
			\end{enumerate}
			
\end{enumerate}


\newpage


\section*{Aufgabe 3}
\begin{enumerate}[label=(\alph*)]
	\item	\begin{enumerate}[label=(\roman*)]
				\item	\textit{Die maximale Höhe des Baums ist $2$.} bzw. \textit{Jeder Pfad von der Wurzel hat eine maximale Tiefe von $2$.}\\
						Im Beispielbaum gilt dieser Satz, dieser hat eine Höhe von $2$. $\mathcal{T} \models \psi_1$
				
				\item	\textit{Jeder Knoten hat entweder $2$ oder keine Kinder/Kanten.}\\
						Im Beispielbaum gilt dieser Satz nicht, denn der Knoten $v_1$ hat nur das \textit{Kind} $v_3$. $\mathcal{T} \not\models \psi_2$
			\end{enumerate}
	
	\item	\textit{Es gibt einen Pfad von der Wurzel ausgehend der Länge $n$.}\\
			Im Beispielbaum gilt dieser Satz nur für $0 \leq n \leq 2, n \in \mathbb{N}$, also nur für $n \in \{1,2\}$.
	
	\item	$$\varphi(x) \coloneqq \forall y (\neg Eyx)$$
\end{enumerate}


%\newpage


\section*{Aufgabe 4}
\begin{enumerate}[label=(\alph*)]
	\item	\begin{enumerate}[label=(\roman*)]
				\item	$$\psi_1(x) \coloneqq \forall y (x \circ y = y)$$
				\item	$$\psi_2(x) \coloneqq \forall y ((x \simeq y) \impl (x = y))$$
			\end{enumerate}
	
	\item	$$\psi_3(x,y) \coloneqq \exists z (x \circ z = y)$$
	\item	Hilffunktion $f_1(x) \coloneqq \exists x_1 \exists x_2 (x_1 \neq x_2 \land x_1 \simeq x \land x_2 \simeq x \land \forall y_1 ((y \simeq x) \impl (y = x_1 \lor y=x_2))$ besagt, dass $x \in \Sigma$.
			
			$$\psi_4(x) \coloneqq ???$$
	\item	$${\psi_5}_n(x) \coloneqq \exists x_1 \dots \exists x_n ((x_1 \circ \dots \circ x_n = x) \impl (\bigwedge_{i=1}^{n} (f_1(x_i) \lor \psi_1(x_i))))$$
	
	\item	$$\psi_6(x) \coloneqq ???$$
\end{enumerate}


\newpage


\section*{Aufgabe 5}

\end{document}
