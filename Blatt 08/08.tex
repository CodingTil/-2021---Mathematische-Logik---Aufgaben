\documentclass[a4paper,11pt]{article}

% Layout
\usepackage[a4paper, left=3cm, right=3cm, top=2cm, bottom=3cm]{geometry} % kleinere Ränder
\usepackage{parskip}

% Umlaute in der Datei erlauben, auf Deutsch umstellen
\usepackage[T1]{fontenc}
\usepackage{lmodern}
\usepackage[utf8]{inputenc}
\usepackage[english, ngerman]{babel}
%\usepackage[english]{babel} % for submissions in ENGLISH

% Mathesymbole und Ähnliches
\usepackage{amsmath}
\usepackage{mathtools}
\usepackage{amssymb}
\usepackage{microtype}
\usepackage{stmaryrd}
\usepackage{enumitem}
\usepackage{bussproofs}

% Grafiken und PDFs einfügen
\usepackage{graphicx}
\usepackage{pdfpages}

% PDF-Tools
\usepackage[hidelinks, unicode]{hyperref}

%%%%%%%%%%%%%%%%%%%%%%%%%%%%%%%%%%%%%%%%
% TIKZ-EINSTELLUNGEN FÜR SPIELE
%%%%%%%%%%%%%%%%%%%%%%%%%%%%%%%%%%%%%%%%

% Abbildungen
\usepackage{tikz}

% einige Bibliotheken
\usetikzlibrary{calc, tikzmark}
\usetikzlibrary{positioning}
\usetikzlibrary{arrows, arrows.meta}
\usetikzlibrary{decorations.markings}

% macht Linien und Pfeile dicker
\tikzset{every picture/.style={thick, >={Latex[round, length=2.5mm, width=2.5mm]}}}
\tikzset{every path/.style={shorten <= 2pt, shorten >= 2pt}}

% Styles für die Knoten der Verifiziererin und des Falsifizierers
\tikzstyle{MCverifier} = [draw, rectangle, rounded corners=1em, semithick, minimum height=2em]
\tikzstyle{MCfalsifier} = [draw, rectangle, thick, minimum height=2em]

%%%%%%%%%%%%%%%%%%%%%%%%%%%%%%%%%%%%%%%%

% Reelle, Natürliche, Ganze, Rationale Zahlen
\newcommand{\R}{\ensuremath{\mathbb{R}}}
\newcommand{\N}{\ensuremath{\mathbb{N}}}
\newcommand{\Z}{\ensuremath{\mathbb{Z}}}
\newcommand{\Q}{\ensuremath{\mathbb{Q}}}

% Fraktur für Strukturen
\newcommand{\A}{\ensuremath{\mathfrak A}}
\newcommand{\B}{\ensuremath{\mathfrak B}}
\newcommand{\I}{\ensuremath{\mathfrak I}}

% Makros für logische Operatoren
\newcommand{\xor}{\ensuremath{\oplus}} % exklusives oder
\newcommand{\impl}{\ensuremath{\rightarrow}} % logische Implikation

% Meistens ist \varphi schöner als \phi, genauso bei \theta
\renewcommand{\phi}{\varphi}
\renewcommand{\theta}{\vartheta}

% Aufzählungen anpassen (alternativ: \arabic, \alph)
\renewcommand{\labelenumi}{(\roman{enumi})}

% BITTE NICHT ÄNDERN: interne Kommandos für die Informationen (Blattnummer, Gruppe, ...)
% PLEASE DO NOT EDIT THIS SECTION

\newcommand{\printsheet}{?}
\newcommand{\sheet}[1]{%
\renewcommand{\printsheet}{#1}%
}

\newcommand{\printgroup}{?}
\newcommand{\group}[1]{%
\renewcommand{\printgroup}{#1}%
}

\newcommand{\printmembers}{}
\newcommand{\printmember}[3]{{#2} {#3} & {#1} \\}

\newcommand{\pdfmembers}{}
\newcommand{\pdfmember}[3]{{#1} {#3}, {#2}; }

\newcommand{\member}[3]{%
\expandafter\renewcommand\expandafter\printmembers\expandafter{\printmembers\printmember{#1}{#2}{#3}}%
\expandafter\renewcommand\expandafter\pdfmembers\expandafter{\pdfmembers\pdfmember{#1}{#2}{#3}}%
}

\AtBeginDocument{\hypersetup{
    pdftitle = {Übungsblatt \printsheet},
    pdfauthor = {Abgabegruppe \printgroup: \pdfmembers},
    pdfsubject = {Mathematische Logik}
}}

% <-------- HIER alle Informationen eintragen ========================
% enter your information here
\sheet{08} % Nummer des Blatts / number of exercise sheet
\group{55} % Gruppennummer der Abgabegruppe in Moodle / group number from Moodle
% Die Gruppennummer erscheint NICHT auf dem Blatt (nur in den PDF-Metadaten).
% The group number does NOT appear on the sheet (check the PDF meta data).

% alle Gruppenmitglieder in der Form \member{Matrikelnummer}{Vorname}{Nachname}
% group members are entered as \member{matriculation number}{first name}{last name}
\member{405401}{Marc}{Ludevid}
\member{405409}{Andrés}{Montoya}
\member{405959}{Til}{Mohr}
%\member{999999}{Viertes}{Mitglied}

\begin{document}

% Platz für die Punktetabelle und Kommentare
\hfill
\begin{Form}
\begin{tabular}{c}
\\
Gesamtpunkte: \\[2mm]
\TextField[name=points, width=20mm, align=1, bordercolor={0 0 0}]{} \\
\\
\end{tabular}
\end{Form}

% Kopfzeile
{\raggedright
\begin{tabular}{l}
    MaLo \\
    SS 2021 \\
    \today{} \\
\end{tabular}}
\hfill
{\Large Übungsblatt \printsheet}
\hfill
\begin{tabular}{l l}
\printmembers
\end{tabular}
\hrule


% <-------- HIER beginnt die Lösung ========================
% your SOLUTION starts here

\section*{Aufgabe 1}
E-Test

\section*{Aufgabe 2}
\begin{enumerate}[label=(\alph*)]
	\item	Folgt aus Skript Lemma 3.14: Eine Theorie ist genau dann vollständig, wenn alle ihre Modelle elementar äquivalent sind. Aus der Aufgabenstellung folgt, dass alle Modelle von der Theorie elementar äquivalent zu $(\mathbb{N}, \cdot, <)$ sind und somit gilt: alle Modelle sind elementar äquivalent und deswegen ist die Theorie vollständig.
	\item	Nach Skript Definition 3.12: die Theorie von jeder Struktur \A ist "offensichtlich" vollständig, also auch $\operatorname{Th}((\mathbb{C}, 0, 1, +, \cdot))$
	\item	\begin{enumerate}[label=(\roman*)]
				\item	
				\item	
			\end{enumerate}
	\item	Alle zu $(\mathbb{Z}, <)$ isomorphen Strukturen sind implizit auch elementar äquivalent. Deswegen folgt wie in a), dass die Theorie vollständig sein muss.
\end{enumerate}


\newpage


\section*{Aufgabe 3}
\begin{enumerate}[label=(\alph*)]
	\item	$\mathfrak{A} \coloneqq (\{q \mid q \in \mathbb{Q}, 0 \leq q \leq 1\}, <)$ \\
			$\mathfrak{B} \coloneqq (\{r \mid r \in \mathbb{R}, 0 \leq r \leq 1\}, <)$
	\item	
	\item	
\end{enumerate}


\newpage


\section*{Aufgabe 4}
\begin{enumerate}[label=(\alph*)]
	\item	$m=4$: \\
			Strategie des Herausforderers für $m=4$:\\
			Wähle $v$. Um die Kantenrelation (Selbstkante) nicht zu verletzen, muss die Duplikatorin nun $6$ auswählen. Anschließend wählt der Herausforderer $p$ aus. Da es eine Kante zwischen $v$ und $p$ gibt, muss $p$ auf einen Knoten abgebildet werden, sodass dieser auch mit $6$ verbunden ist. Die Duplikatorin wählt also $1$ aus. Nun wählt der Herausforderer $r$ aus. $r$ steht nicht in der Kantenrelation bezüglich $p$ oder $v$, folglich muss die Duplikatorin entweder $3$ oder $4$ auswählen. Unabhängig davon, was gewählt wird, wählt der Herausforderer nun $t$, welcher ebenfalls keine Kante zu den anderen Knoten hat. Die Duplikatorin kann nun aber nichtmehr einen Knoten auswählen.\\
			
			Strategie der Duplikatorin für $m=3$:\\
			Egal welche Knoten der Herausforder wählt, kann die Duplikatorin immer einen passenden Knoten finden, sodass die Kantenrelationen zwischen $\mathfrak{A}$ und $\mathfrak{B}$ übereinstimmen.
	\item	$m=2$:\\
			Strategie des Herausforderers für $m=2$:\\
			Wähle $\{0,1\}$, dann ist es egal welchen knoten $k \in \mathcal{P}(\mathbb{Z})$ die Duplikatorin wählt weil eine beliebige Obermenge von k von dem Herausforderer im zweiten Schritt ausgewählt werden kann und dies von der Duplikatorin in $\mathcal{P}(\{0,1\})$ nicht nachmachen kann.\\
			
			Strategie der Duplikatorin für $m=1$:\\
			Egal was gewählt wird einfach irgend ein Element aus der anderen Potenzmenge wählen.
	\item	$m = 2$:\\
			$\phi = \exists x \exists y (M(x,x,x) \land M(y,y,x) \land x \not = y)$\\
			Für $\mathbb{Z}$ geht das für $x = 1, y = -1$.\\
			Für $\mathbb{N}$ geht das nicht, denn $x$ müsste $0$ oder $1$ sein aber es gibt keine Zahl $y$ unterschiedlich zu $x$ die multipliziert mit sich selber $x$ ergibt.\\
			
			Strategie der Duplikatorin für $m=1$:\\
			Falls der Herausforderer $0$ oder $1$ wählt, nehme die selbe Zahl im anderen Universum. Andernfalls wähle die $2$ im entsprechenden Universum.
\end{enumerate}


\newpage


\section*{Aufgabe 5}
Sei $B$ ein unendlicher ungerichtete Graph, sodass es einen Knoten $k$ gibt, der zu allen anderen Knoten verbunden ist, ohne dass diese anderen unter sich verbunden sind (Sterförmig). Dieser Graph ist offensichtlich nicht in $\mathcal{K}$, weil $k$ einen unendlichen Grad hat.\\
Für beliebiges $m \in \mathbb{N}$ sei $A_m$ ein Graph mit $m+1$ Knoten. $A_m$ hat ebenfalls einen zentralen Knoten $k'$ und weitere $m$ Peripherieknoten, die alle mit dem zentralen Knoten verbunden sind, ohne untereinander verbunden zu sein. Egal welche $m$ Knoten der Herausforderer in einem der beiden Graphen auswählt, die Duplikatorin kann immer einen entsprechenden Peripherieknoten bzw. den zentralen Knoten im anderen Graphen finden und trotzdem die Kantenrelation nicht verletzen.\\
Da alle $A_m$ in der Klasse $\mathcal{K}$ liegen und alle $A_m$ $m$-equivalent zu $B$ sind, ist $\mathcal{K}$ nicht FO-axiomatisierbar.


\end{document}
