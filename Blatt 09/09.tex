\documentclass[a4paper,11pt]{article}

% Layout
\usepackage[a4paper, left=3cm, right=3cm, top=2cm, bottom=3cm]{geometry} % kleinere Ränder
\usepackage{parskip}
\usepackage{rotating}

% Umlaute in der Datei erlauben, auf Deutsch umstellen
\usepackage[T1]{fontenc}
\usepackage{lmodern}
\usepackage[utf8]{inputenc}
\usepackage[english, ngerman]{babel}
%\usepackage[english]{babel} % for submissions in ENGLISH

% Mathesymbole und Ähnliches
\usepackage{amsmath}
\usepackage{mathtools}
\usepackage{amssymb}
\usepackage{microtype}
\usepackage{stmaryrd}
\usepackage{enumitem}
\usepackage{bussproofs}

% Grafiken und PDFs einfügen
\usepackage{graphicx}
\usepackage{pdfpages}

% PDF-Tools
\usepackage[hidelinks, unicode]{hyperref}

%%%%%%%%%%%%%%%%%%%%%%%%%%%%%%%%%%%%%%%%
% TIKZ-EINSTELLUNGEN FÜR SPIELE
%%%%%%%%%%%%%%%%%%%%%%%%%%%%%%%%%%%%%%%%

% Abbildungen
\usepackage{tikz}

% einige Bibliotheken
\usetikzlibrary{calc, tikzmark}
\usetikzlibrary{positioning}
\usetikzlibrary{arrows, arrows.meta}
\usetikzlibrary{decorations.markings}

% macht Linien und Pfeile dicker
\tikzset{every picture/.style={thick, >={Latex[round, length=2.5mm, width=2.5mm]}}}
\tikzset{every path/.style={shorten <= 2pt, shorten >= 2pt}}

% Styles für die Knoten der Verifiziererin und des Falsifizierers
\tikzstyle{MCverifier} = [draw, rectangle, rounded corners=1em, semithick, minimum height=2em]
\tikzstyle{MCfalsifier} = [draw, rectangle, thick, minimum height=2em]

%%%%%%%%%%%%%%%%%%%%%%%%%%%%%%%%%%%%%%%%

% Reelle, Natürliche, Ganze, Rationale Zahlen
\newcommand{\R}{\ensuremath{\mathbb{R}}}
\newcommand{\N}{\ensuremath{\mathbb{N}}}
\newcommand{\Z}{\ensuremath{\mathbb{Z}}}
\newcommand{\Q}{\ensuremath{\mathbb{Q}}}

% Fraktur für Strukturen
\newcommand{\A}{\ensuremath{\mathfrak A}}
\newcommand{\B}{\ensuremath{\mathfrak B}}
\newcommand{\I}{\ensuremath{\mathfrak I}}

% Makros für logische Operatoren
\newcommand{\xor}{\ensuremath{\oplus}} % exklusives oder
\newcommand{\impl}{\ensuremath{\rightarrow}} % logische Implikation

% Meistens ist \varphi schöner als \phi, genauso bei \theta
\renewcommand{\phi}{\varphi}
\renewcommand{\theta}{\vartheta}

% Aufzählungen anpassen (alternativ: \arabic, \alph)
\renewcommand{\labelenumi}{(\roman{enumi})}

% BITTE NICHT ÄNDERN: interne Kommandos für die Informationen (Blattnummer, Gruppe, ...)
% PLEASE DO NOT EDIT THIS SECTION

\newcommand{\printsheet}{?}
\newcommand{\sheet}[1]{%
\renewcommand{\printsheet}{#1}%
}

\newcommand{\printgroup}{?}
\newcommand{\group}[1]{%
\renewcommand{\printgroup}{#1}%
}

\newcommand{\printmembers}{}
\newcommand{\printmember}[3]{{#2} {#3} & {#1} \\}

\newcommand{\pdfmembers}{}
\newcommand{\pdfmember}[3]{{#1} {#3}, {#2}; }

\newcommand{\member}[3]{%
\expandafter\renewcommand\expandafter\printmembers\expandafter{\printmembers\printmember{#1}{#2}{#3}}%
\expandafter\renewcommand\expandafter\pdfmembers\expandafter{\pdfmembers\pdfmember{#1}{#2}{#3}}%
}

\AtBeginDocument{\hypersetup{
    pdftitle = {Übungsblatt \printsheet},
    pdfauthor = {Abgabegruppe \printgroup: \pdfmembers},
    pdfsubject = {Mathematische Logik}
}}

% <-------- HIER alle Informationen eintragen ========================
% enter your information here
\sheet{09} % Nummer des Blatts / number of exercise sheet
\group{55} % Gruppennummer der Abgabegruppe in Moodle / group number from Moodle
% Die Gruppennummer erscheint NICHT auf dem Blatt (nur in den PDF-Metadaten).
% The group number does NOT appear on the sheet (check the PDF meta data).

% alle Gruppenmitglieder in der Form \member{Matrikelnummer}{Vorname}{Nachname}
% group members are entered as \member{matriculation number}{first name}{last name}
\member{405401}{Marc}{Ludevid}
\member{405409}{Andrés}{Montoya}
\member{405959}{Til}{Mohr}
%\member{999999}{Viertes}{Mitglied}

\begin{document}

% Platz für die Punktetabelle und Kommentare
\hfill
\begin{Form}
\begin{tabular}{c}
\\
Gesamtpunkte: \\[2mm]
\TextField[name=points, width=20mm, align=1, bordercolor={0 0 0}]{} \\
\\
\end{tabular}
\end{Form}

% Kopfzeile
{\raggedright
\begin{tabular}{l}
    MaLo \\
    SS 2021 \\
    \today{} \\
\end{tabular}}
\hfill
{\Large Übungsblatt \printsheet}
\hfill
\begin{tabular}{l l}
\printmembers
\end{tabular}
\hrule


% <-------- HIER beginnt die Lösung ========================
% your SOLUTION starts here

\section*{Aufgabe 1}
E-Test

\section*{Aufgabe 2}


\newpage


\section*{Aufgabe 3}
\begin{turn}{90}
\begin{minipage}{\paperwidth}
\begin{prooftree}
			\AxiomC{$c_1 < c_2, c_2 < c_1 \Rightarrow c_1 < c_1, c_1 < c_2$}
			\AxiomC{$c_1 < c_2, c_2 < c_1 \Rightarrow c_1 < c_1, c_2 < c_1$}
		
		\LeftLabel{$(\Rightarrow \land)$}
		\BinaryInfC{$c_1 < c_2, c_2 < c_1 \Rightarrow c_1 < c_1, c_1 < c_2 \land c_2 < c_1$}		
		
		\AxiomC{$c_1 < c_1, c_1 < c_2, c_2 < c_1 \Rightarrow c_1 < c_1$}
	
	\LeftLabel{$(\impl \Rightarrow)$}
	\BinaryInfC{$(c_1 < c_2 \land c_2 < c_1) \impl c_1 < c_1, c_1 < c_2, c_2 < c_1 \Rightarrow c_1 < c_1$}
	
	\LeftLabel{$(\land \Rightarrow)$}
	\UnaryInfC{$(c_1 < c_2 \land c_2 < c_1) \impl c_1 < c_1, c_1 < c_2 \land c_2 < c_1 \Rightarrow c_1 < c_1$}
	
	\LeftLabel{$(\neg \Rightarrow)$}
	\UnaryInfC{$\neg c_1 < c_1, (c_1 < c_2 \land c_2 < c_1) \impl c_1 < c_1 \Rightarrow \neg (c_1 < c_2 \land c_2 < c_1)$}
	
	\LeftLabel{$(\Rightarrow \forall)$}
	\UnaryInfC{$\neg c_1 < c_1, (c_1 < c_2 \land c_2 < c_1) \impl c_1 < c_1 \Rightarrow \forall y \neg (c_1 < y \land y < c_1)$}
	
	\LeftLabel{$(\Rightarrow \forall)$}
	\UnaryInfC{$\neg c_1 < c_1, (c_1 < c_2 \land c_2 < c_1) \impl c_1 < c_1 \Rightarrow \forall x \forall y \neg (x < y \land y < x)$}
	
	\LeftLabel{$(\forall \Rightarrow)$}
	\UnaryInfC{$\neg c_1 < c_1, \forall z ((c_1 < c_2 \land c_2 < z) \impl c_1 < z) \Rightarrow \forall x \forall y \neg (x < y \land y < x)$}
	
	\LeftLabel{$(\forall \Rightarrow)$}
	\UnaryInfC{$\neg c_1 < c_1, \forall y \forall z ((c_1 < y \land y < z) \impl c_1 < z) \Rightarrow \forall x \forall y \neg (x < y \land y < x)$}
	
	\LeftLabel{$(\forall \Rightarrow)$}
	\UnaryInfC{$\neg c_1 < c_1, \forall x \forall y \forall z ((x < y \land y < z) \impl x < z) \Rightarrow \forall x \forall y \neg (x < y \land y < x)$}
	
	\LeftLabel{$(\forall \Rightarrow)$}
	\UnaryInfC{$\forall x (\neg x < x), \forall x \forall y \forall z ((x < y \land y < z) \impl x < z) \Rightarrow \forall x \forall y \neg (x < y \land y < x)$}
\end{prooftree}
\end{minipage}
\end{turn}

Da alle Blätter Axiome sind, ist die Sequenz gültig.

\newpage


\section*{Aufgabe 4}
\begin{enumerate}[label=(\alph*)]
	\item	\begin{enumerate}[label=(\roman*)]
				\item	
				\item	Da $g$ nicht in $\Gamma \cup \Delta \cup \{\varphi\}$ vorkommt, kann man $g$ so wählen, dass $g(x)$ genau dem $y$ aus der Konklusion entspricht. Gilt die Prämisse, so folglich auch die Konklusion, da wir $y$ mit dem $g(x)$ ``ersetzen'' können.
			\end{enumerate}
	\item	\begin{enumerate}[label=(\roman*)]
				\item	
				\item	
			\end{enumerate}
\end{enumerate}


\newpage


\section*{Aufgabe 5}
\begin{enumerate}[label=(\alph*)]
	\item	\begin{enumerate}[label=(\roman*)]
				\item	Damit $\sim$ eine Äquivalenzrelation auf $\mathcal{P}(\mathbb{N})$ ist, muss sie reflexiv, symmetrisch und transitiv sein.\\
						
						Für jedes $A \in \mathcal{P}(\mathbb{N})$ gilt offensichtlich $A \sim A$, da $\vert A \vert = \vert A \vert$. $\sim$ ist also reflexiv.\\
						
						Seien $A,B \in \mathcal{P}(\mathbb{N})$. Angenommen es gilt $A \sim B$. Dann gilt $\vert A \vert = \vert B \vert$, welches äquivalent ist zu $\vert B \vert = \vert A \vert$. Folglich ist $\sim$ symmetrisch.\\
						
						Seien $A,B,C \in \mathcal{P}(\mathbb{N})$. Angenommen es gilt sowohl $A \sim B$ als auch $B \sim C$. Dann muss ja gelten, dass $\vert A \vert = \vert B \vert$ und $\vert B \vert = \vert C \vert$. Insbesondere gilt dann auch $\vert A \vert = \vert C \vert$. Folglich muss dann auch $A \sim C$ gelten. $\sim$ ist also transitiv.\\
						
						Damit ist $\sim$ eine Äquivalenzrelation auf $\mathcal{P}(\mathbb{N})$.
				\item	Damit $\sim$ auf $\mathfrak{A}$ eine Kongruenzrelation ist, muss unter anderem $\cup$ mit $\sim$ verträglich sein.\\
						
						Seien $A_1 \coloneqq \{1,2\}, A_2 \coloneqq \{3,4\}, B_1 \coloneqq \{5,6\}, B_2 \coloneqq \{6,7\}$.\\
						Es gilt offensichtlich $A_1 \sim B_1$ und $A_2 \sim B_2$. Jedoch gilt $A_1 \cup A_2 \sim B_1 \cup B_2$ nicht, da $\vert A_1 \cup A_2 \vert = \vert \{1,2,3,4\} \vert = 4 \neq 3 = \vert \{5,6,7\} \vert = \vert B_1 \cup B_2 \vert$.\\
						
						Damit ist $\sim$ keine Kongruenzrelation auf $\mathfrak{A}$.
			\end{enumerate}
	\item	\begin{enumerate}[label=(\roman*)]
				\item	Seien $A_1, A_2, B_1, B_2 \in \mathcal{P}(\mathbb{N})$ und gelte $A_1 \sim_2 B_1, A_2 \sim_2 B_2$. Wir müssen nun zeigen, dass $\cup$ und $\cap$ mit $\sim_2$ verträglich sind.\\
						
						Es gelte $A_1 \cup A_2 \sim_2 B_1 \cup B_2$, denn:
						\begin{align*}
							((A_1 \cup A_2) \cap 2\mathbb{N}) &= (A_1 \cap 2\mathbb{N}) \cup (A_2 \cap 2\mathbb{N}) \\
								&\overset{*}{=} (B_1 \cap 2\mathbb{N}) \cup (B_2 \cap 2\mathbb{N}) \\
								&= ((B_1 \cup B_2) \cap 2\mathbb{N})
						\end{align*}
						$*$ gilt, da eben $A_1 \sim_2 B_1, A_2 \sim_2 B_2$.\\
						Folglich ist $\cup$ mit $\sim$ verträglich.\\
						
						Es gelte $A_1 \cap A_2 \sim_2 B_1 \cap B_2$, denn:
						\begin{align*}
							((A_1 \cap A_2) \cap 2\mathbb{N}) &\overset{*}{=} (A_1 \cap A_2 \cap 2\mathbb{N} \cap 2\mathbb{N}) \\
								&= (A_1 \cap 2\mathbb{N}) \cap (A_2 \cap 2\mathbb{N}) \\
								&\overset{**}{=} (B_1 \cap 2\mathbb{N}) \cap (B_2 \cap 2\mathbb{N}) \\
								&\overset{*}{=} (B_1 \cap B_2 \cap 2\mathbb{N} \cap 2\mathbb{N}) \\
								&= ((B_1 \cap B_2) \cap 2\mathbb{N})
						\end{align*}
						$*$ gilt, da eben $A_1 \sim_2 B_1, A_2 \sim_2 B_2$.\\
						$**$ gilt, da offensichtlich für jede Menge $X$ gilt: $X = X \cap X$.\\
						Folglich ist $\cap$ mit $\sim$ verträglich.\\
						
						Also ist $\sim_2$ eine Kongruenzrelation auf $\mathfrak{A}$.
				\item	
			\end{enumerate}
\end{enumerate}


\end{document}
