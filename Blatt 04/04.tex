\documentclass[a4paper,11pt]{article}

% Layout
\usepackage[a4paper, left=3cm, right=3cm, top=2cm, bottom=3cm]{geometry} % kleinere Ränder
\usepackage{parskip}

% Umlaute in der Datei erlauben, auf Deutsch umstellen
\usepackage[T1]{fontenc}
\usepackage{lmodern}
\usepackage[utf8]{inputenc}
\usepackage[english, ngerman]{babel}
%\usepackage[english]{babel} % for submissions in ENGLISH

% Mathesymbole und Ähnliches
\usepackage{amsmath}
\usepackage{mathtools}
\usepackage{amssymb}
\usepackage{microtype}
\usepackage{stmaryrd}
\usepackage{enumitem}
\usepackage{bussproofs}

% Grafiken und PDFs einfügen
\usepackage{graphicx}
\usepackage{pdfpages}

% PDF-Tools
\usepackage[hidelinks, unicode]{hyperref}

% Abbildungen
\usepackage{tikz}
\usetikzlibrary{arrows,calc}

% Reelle, Natürliche, Ganze, Rationale Zahlen
\newcommand{\R}{\ensuremath{\mathbb{R}}}
\newcommand{\N}{\ensuremath{\mathbb{N}}}
\newcommand{\Z}{\ensuremath{\mathbb{Z}}}
\newcommand{\Q}{\ensuremath{\mathbb{Q}}}

% Fraktur für Strukturen
\newcommand{\A}{\ensuremath{\mathfrak A}}
\newcommand{\B}{\ensuremath{\mathfrak B}}
\newcommand{\I}{\ensuremath{\mathfrak I}}

% Makros für logische Operatoren
\newcommand{\xor}{\ensuremath{\oplus}} % exklusives oder
\newcommand{\impl}{\ensuremath{\rightarrow}} % logische Implikation

% Meistens ist \varphi schöner als \phi, genauso bei \theta
\renewcommand{\phi}{\varphi}
\renewcommand{\theta}{\vartheta}

% Aufzählungen anpassen (alternativ: \arabic, \alph)
\renewcommand{\labelenumi}{(\roman{enumi})}

% BITTE NICHT ÄNDERN: interne Kommandos für die Informationen (Blattnummer, Gruppe, ...)
% PLEASE DO NOT EDIT THIS SECTION

\newcommand{\printsheet}{?}
\newcommand{\sheet}[1]{%
\renewcommand{\printsheet}{#1}%
}

\newcommand{\printgroup}{?}
\newcommand{\group}[1]{%
\renewcommand{\printgroup}{#1}%
}

\newcommand{\printmembers}{}
\newcommand{\printmember}[3]{{#2} {#3} & {#1} \\}

\newcommand{\pdfmembers}{}
\newcommand{\pdfmember}[3]{{#1} {#3}, {#2}; }

\newcommand{\member}[3]{%
\expandafter\renewcommand\expandafter\printmembers\expandafter{\printmembers\printmember{#1}{#2}{#3}}%
\expandafter\renewcommand\expandafter\pdfmembers\expandafter{\pdfmembers\pdfmember{#1}{#2}{#3}}%
}

\AtBeginDocument{\hypersetup{
    pdftitle = {Übungsblatt \printsheet},
    pdfauthor = {Abgabegruppe \printgroup: \pdfmembers},
    pdfsubject = {Mathematische Logik}
}}

% <-------- HIER alle Informationen eintragen ========================
% enter your information here
\sheet{04} % Nummer des Blatts / number of exercise sheet
\group{55} % Gruppennummer der Abgabegruppe in Moodle / group number from Moodle
% Die Gruppennummer erscheint NICHT auf dem Blatt (nur in den PDF-Metadaten).
% The group number does NOT appear on the sheet (check the PDF meta data).

% alle Gruppenmitglieder in der Form \member{Matrikelnummer}{Vorname}{Nachname}
% group members are entered as \member{matriculation number}{first name}{last name}
\member{405401}{Marc}{Ludevid}
\member{405409}{Andrés}{Montoya}
\member{405959}{Til}{Mohr}
%\member{999999}{Viertes}{Mitglied}

\begin{document}

% Platz für die Punktetabelle und Kommentare
\hfill
\begin{Form}
\begin{tabular}{c}
\\
Gesamtpunkte: \\[2mm]
\TextField[name=points, width=20mm, align=1, bordercolor={0 0 0}]{} \\
\\
\end{tabular}
\end{Form}

% Kopfzeile
{\raggedright
\begin{tabular}{l}
    MaLo \\
    SS 2021 \\
    \today{} \\
\end{tabular}}
\hfill
{\Large Übungsblatt \printsheet}
\hfill
\begin{tabular}{l l}
\printmembers
\end{tabular}
\hrule


% <-------- HIER beginnt die Lösung ========================
% your SOLUTION starts here

\section*{Aufgabe 1}
E-Test

\section*{Aufgabe 2}
\begin{enumerate}[label=(\alph*)]
	\item	Falls $\varphi_1$ unerfüllbar ist, dann muss $\varphi_1 \Rightarrow \emptyset$ gültig sein.
			
			\begin{prooftree}
						\AxiomC{$X \lor Y \Rightarrow X, Y$}
							
							\RightLabel{$(*)$}
							\AxiomC{$X, Z \Rightarrow Y$}
							\AxiomC{$Y, Z \Rightarrow Y$}
							
						\LeftLabel{$(\lor \Rightarrow)$}
						\BinaryInfC{$X \lor Y, Z \Rightarrow Y$}
					
					\LeftLabel{$(\rightarrow \Rightarrow)$}
					\BinaryInfC{$X \lor Y, X \impl Z \Rightarrow Y$}
					
					\AxiomC{$X \lor Y, X \impl Z \Rightarrow \neg Z$}
					
				\LeftLabel{$(\Rightarrow \land)$}
				\BinaryInfC{$X \lor Y, X \impl Z \Rightarrow Y \land \neg Z$}
				
				\LeftLabel{$(\neg \Rightarrow)$}
				\UnaryInfC{$\neg (Y \land \neg Z), X \lor Y, X \impl Z \Rightarrow \emptyset$}
				
				\LeftLabel{$(\land \Rightarrow)$}
				\UnaryInfC{$\neg (Y \land \neg Z), (X \lor Y) \land (X \impl Z) \Rightarrow \emptyset$}
				
				\LeftLabel{$(\land \Rightarrow)$}
				\UnaryInfC{$\neg (Y \land \neg Z) \land ((X \lor Y) \land (X \impl Z)) \Rightarrow \emptyset$}
			\end{prooftree}
			
			$(*)$ liefert uns jedoch die falsifizierende Interpretation $\mathfrak{I}: X,Z \mapsto 1, Y \mapsto 0$, die also ein Modell für $\varphi_1$ ist.
	
	\item	Falls $\varphi_2$ eine Tautologie ist, muss $\emptyset \Rightarrow \varphi_2$ gültig sein.
	
			\begin{prooftree}
						\AxiomC{$C \Rightarrow C, A, B$}
						
						\AxiomC{$B \Rightarrow C, A, B$}					
						
					\LeftLabel{$(\lor \Rightarrow)$}
					\BinaryInfC{$C \lor B \Rightarrow C, A, B$}					
					
					\LeftLabel{$(\Rightarrow \lor)$}
					\UnaryInfC{$C \lor B \Rightarrow C, A \lor B$}
					
					\AxiomC{$C \lor B, C \Rightarrow C$}
				
				\LeftLabel{$(\rightarrow \Rightarrow)$}
				\BinaryInfC{$(A \lor B) \impl C, C \lor B \Rightarrow C$}
				
				\LeftLabel{$(\neg \Rightarrow)$}
				\UnaryInfC{$(A \lor B) \impl C, C \lor B, \neg C \Rightarrow C$}
				
				\LeftLabel{$(\Rightarrow \neg)$}
				\UnaryInfC{$(A \lor B) \impl C, C \lor B \Rightarrow C, \neg\neg C$}
				
				\LeftLabel{$(\Rightarrow \lor)$}
				\UnaryInfC{$(A \lor B) \impl C, C \lor B \Rightarrow C \lor \neg\neg C$}
				
				\LeftLabel{$(\land \Rightarrow)$}
				\UnaryInfC{$((A \lor B) \impl C) \land (C \lor B) \Rightarrow C \lor \neg\neg C$}
				
				\LeftLabel{$(\Rightarrow \rightarrow)$}
				\UnaryInfC{$\emptyset \Rightarrow (((A \lor B) \impl C) \land (C \lor B)) \impl (C \lor \neg\neg C)$}
			\end{prooftree}
			
			Da alle Blätter Axiome sind, ist die Sequenz bewiesen und $\varphi_2$ also eine Tautologie.
\end{enumerate}


\newpage


\section*{Aufgabe 3}
\begin{enumerate}[label=(\alph*)]
	\item	
	\item	
	
			% Unsicher, wollte aber noch nicht löschen :)
			%Wenn die erste Prämisse gültig ist, dann muss $\Gamma, \neg \psi, \neg \vartheta \Rightarrow \Delta$ auch gültig sein, denn es kann kein Modell der linken Seite geben, die $\psi$ oder $\vartheta$ erfüllt. Folglich muss auch $\Gamma \Rightarrow \Delta$ gültig sein, oder $\Gamma, \neg \psi, \neg \vartheta$ hat kein Modell.\\
			%Angenommen $\Gamma \Rightarrow \Delta$ ist gültig und die zweite Prämisse auch. Dann ist die Konklusion offensichtlich auch gültig.\\
			%Angenommen $\Gamme \Rightarrow \Delta$ ist nicht gültig, aber die zweite Prämisse wohl. Dann ist $\Gamma, \varphi \Rightarrow \psi$ oder $\Gamma, \varphi \Rightarrow \vartheta$ gültig. Gilt ersteres, dann ist die Konklusion gültig .....
			%Die Schlussregel ist also korrekt.
\end{enumerate}

\newpage


\section*{Aufgabe 4}
\begin{enumerate}[label=(\alph*)]
	\item	\begin{prooftree}
			\AxiomC{$\Gamma \Rightarrow \Delta, \varphi, \psi$}
			\AxiomC{$\Gamma, \varphi, \psi \Rightarrow \Delta$}
			
			\LeftLabel{$(\leftrightarrow \Rightarrow)$}
			\BinaryInfC{$\Gamma, \varphi \leftrightarrow \psi \Rightarrow \Delta$}
			\end{prooftree}
			
			
			\begin{prooftree}
			\AxiomC{$\Gamma, \varphi \Rightarrow \Delta, \psi$}
			\AxiomC{$\Gamma, \psi \Rightarrow \Delta, \varphi$}
			
			\LeftLabel{$(\Rightarrow \leftrightarrow)$}
			\BinaryInfC{$\Gamma \Rightarrow \Delta, \varphi \leftrightarrow \psi$}
			\end{prooftree}
	\item	\begin{enumerate}[label=(\roman*)]
				\item	Sei $\Gamma = \emptyset = \Delta, \psi = x, \vartheta = \neg x$. Die Prämisse $x, \neg x \Rightarrow \emptyset$ ist offensichtlich gültig, da die linke Seite unerfüllbar ist. Die Konklusion ist jedoch ungültig, da $x \lor \neg x$ erfüllbar ist.\\
						Somit ist $(\lor \Rightarrow)'$ nicht korrekt.
						
				\item	Fall 1: Sei $\Gamma \Rightarrow \Delta$ gültig. Dann sind offensichtlich beide Prämissen und die Konklusion gültig.\\
						Fall 2: Sei $\Gamma \Rightarrow \Delta$ nicht gültig. Damit die Prämissen gültig sind, muss $\Gamma \Rightarrow \psi$ und $\Gamma \Rightarrow \vartheta$ gelten. Dann ist ja auch $\Gamma \Rightarrow \psi \lor \vartheta$ gültig.\\
						Damit ist $(\Rightarrow \lor)'$ korrekt.\\
						
						Damit ist der Sequenzenkalkül immer noch korrekt, wenn wir die Benutzung der Schlussregel $(\Rightarrow \lor)'$ erlauben, denn, angenommen es wäre nicht korrekt, würden wir auf einen Widerspruch laufen:\\
						Sei der Sequenzenkalkül also nicht korrekt. Dann gibt es eine nicht-gültige Sequenz, die trotzdem als gültig abgeleitet werden kann. Dann sind die Prämissen von $(\Rightarrow \lor)'$ gültig. Da aber $(\Rightarrow \lor)'$ korrekt ist, muss auch die Konklusion gültig sein. Dies führt zu einem Widerspruch.
						
				\item	Es ist plausibel, dass $\mathfrak{I}_1, \mathfrak{I}_2$ keine falsifizierende Interpretationen sind, da der erweiterte Sequenzenkalkül nicht vollständig ist. Es gibt also Sequenzen, die gültig sind, jedoch nicht als gültig abgeleitet werden:\\
						
						Das erweiterte Sequenzenkalkül ist jedoch nicht korrekt.\\
						Sei $\Gamma = \{x\}, \Delta = \emptyset, \psi = x, \vartheta = \neg x$.
						Dann ist $\Gamma \Rightarrow \Delta, \psi \lor \vartheta$ nach dem unveränderten Sequenzenkalkül offensichtlich gültig:
						\begin{prooftree}
							\AxiomC{$x \Rightarrow x, \neg x$}
							
							\LeftLabel{$(\Rightarrow \lor)$}
							\UnaryInfC{$x \Rightarrow x \lor \neg x$}
						\end{prooftree}
						
						Unter Verwendung von $(\Rightarrow \lor)'$ erhalten wir jedoch eine falsifizierende Interpretation $\mathfrak{I}$:
						\begin{prooftree}
								\AxiomC{$x \Rightarrow x$}
										
									\AxiomC{$x \Rightarrow \emptyset$}										
										
								\LeftLabel{$(\Rightarrow \neg)$}
								\UnaryInfC{$x \Rightarrow \neg x$}
							
							\LeftLabel{$(\Rightarrow \lor)'$}
							\BinaryInfC{$x \Rightarrow x \lor \neg x$}
						\end{prooftree}
						$\mathfrak{I}: x \mapsto 1$
						
						Da wir nun aus einer gültigen Sequenz Ungültigkeit ableiten können, ist das erweiterte Sequenzenkalkül nicht vollständig.
						
						
						Damit beide Interpretationen falsifizierend sind, müsste der erweiterte Sequenzenkalkül vollständig sein.
			\end{enumerate}
\end{enumerate}


\newpage


\section*{Aufgabe 5}
Sei der Baum $T$ folgendermassen definiert:

Jeder Knoten im Baum ist eine Fliesenverlegung $f: (\{0,...,n\}, \{0,...,n\}) \rightarrow \{0,...,m\}$ wobei $n$ der Tiefe der Tiefe des Knotens im Baum entspricht. Für die Wurzel gilt also $f_w: (\{\}, \{\}) \rightarrow \{0,...,m\}$, was die eindeutige Fliesenverlegung für einen Raum der Grösse $0 \times 0$ ist.

Für jeden Knoten mit der Fliesenverlegung $f$ und einer Tiefe $n$ bilden wir die möglichen Kinderknoten $f'$ folgendermassen:

$f': (\{0,...,n,n+1\}, \{0,...,n,n+1\}) \rightarrow \{0,...,m\}$ wobei gelten muss $f'(x,y) = f(x,y)$ für alle $0 \leq x,y \leq n$.

Jeder Kinderknoten ist also eine Erweiterung der Fliesenverlegung auf eine Reihe und eine Spalte mehr.

Für die restlichen $f'(x,y)$ für $x = n+1$ oder $y = n+1$ muss sichergestellt werden, dass die Fliesenverlegungsregeln eingehalten werden. Für $f'(x,y)$ und $0 \leq x,y \leq n$ gelten die Fliesenverlegungsregeln bereits wegen dem induktiven Aufbau des Baums.

Da nicht sichergestellt ist das es so eine Fliesenverlegung $f'$ existiert kann es Knoten geben die keine Kinder haben und somit ein Blatt sind.

Um das Lemma von König anwenden zu können müssen wir noch zeigen, dass alle Knoten endliche Unterknoten haben. Das ist natürlich offensichtlich da jeder Knoten der Tiefe $i$ höchstens $m^{2i + 1}$ Kinderknoten haben kann.

Also können wir das Lemmma von König anwenden. Wenn es also für $m$ verschiedene Fliesen möglich ist für jedes beliebiges $n$ eine Fliesenverlegung $f$ zu finden, dann gibt es im Baum $T$ ebenfalls diese Fliesenverlegung $f$. Aufgrund der Konstruktion hat $f$ die Tiefe $n$ und somit gibt es in $T$ für jedes beliebige $n$ einen Weg $(f_0, f_1, ..., f_n)$ der Länge $n$. Das Lemma von König folgert nun, dass es somit auch einen unendlichen Weg $(f_0, f_1, ...)$ in $T$ gibt. Dauraus folgt, dass es eine Fliesenverlegung $f: (\mathbb{N},\mathbb{N}) \rightarrow \{0,...,m\}$ gibt indem man $f(x,y) = f_{max\{x,y\}}(x,y)$ nutzt.

\end{document}
