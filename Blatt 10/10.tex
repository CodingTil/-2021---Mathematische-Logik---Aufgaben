\documentclass[a4paper,11pt]{article}

% Layout
\usepackage[a4paper, left=3cm, right=3cm, top=2cm, bottom=3cm]{geometry} % kleinere Ränder
\usepackage{parskip}
\usepackage{rotating}

% Umlaute in der Datei erlauben, auf Deutsch umstellen
\usepackage[T1]{fontenc}
\usepackage{lmodern}
\usepackage[utf8]{inputenc}
\usepackage[english, ngerman]{babel}
%\usepackage[english]{babel} % for submissions in ENGLISH

% Mathesymbole und Ähnliches
\usepackage{amsmath}
\usepackage{mathtools}
\usepackage{amssymb}
\usepackage{microtype}
\usepackage{stmaryrd}
\usepackage{enumitem}
\usepackage{bussproofs}

% Grafiken und PDFs einfügen
\usepackage{graphicx}
\usepackage{pdfpages}

% PDF-Tools
\usepackage[hidelinks, unicode]{hyperref}

%%%%%%%%%%%%%%%%%%%%%%%%%%%%%%%%%%%%%%%%
% TIKZ-EINSTELLUNGEN FÜR SPIELE
%%%%%%%%%%%%%%%%%%%%%%%%%%%%%%%%%%%%%%%%

% Abbildungen
\usepackage{tikz}

% einige Bibliotheken
\usetikzlibrary{calc, tikzmark}
\usetikzlibrary{positioning}
\usetikzlibrary{arrows, arrows.meta}
\usetikzlibrary{decorations.markings}

% macht Linien und Pfeile dicker
\tikzset{every picture/.style={thick, >={Latex[round, length=2.5mm, width=2.5mm]}}}
\tikzset{every path/.style={shorten <= 2pt, shorten >= 2pt}}

% Styles für die Knoten der Verifiziererin und des Falsifizierers
\tikzstyle{MCverifier} = [draw, rectangle, rounded corners=1em, semithick, minimum height=2em]
\tikzstyle{MCfalsifier} = [draw, rectangle, thick, minimum height=2em]

%%%%%%%%%%%%%%%%%%%%%%%%%%%%%%%%%%%%%%%%

% Reelle, Natürliche, Ganze, Rationale Zahlen
\newcommand{\R}{\ensuremath{\mathbb{R}}}
\newcommand{\N}{\ensuremath{\mathbb{N}}}
\newcommand{\Z}{\ensuremath{\mathbb{Z}}}
\newcommand{\Q}{\ensuremath{\mathbb{Q}}}

% Fraktur für Strukturen
\newcommand{\A}{\ensuremath{\mathfrak A}}
\newcommand{\B}{\ensuremath{\mathfrak B}}
\newcommand{\I}{\ensuremath{\mathfrak I}}

% Makros für logische Operatoren
\newcommand{\xor}{\ensuremath{\oplus}} % exklusives oder
\newcommand{\impl}{\ensuremath{\rightarrow}} % logische Implikation

% Meistens ist \varphi schöner als \phi, genauso bei \theta
\renewcommand{\phi}{\varphi}
\renewcommand{\theta}{\vartheta}

% Aufzählungen anpassen (alternativ: \arabic, \alph)
\renewcommand{\labelenumi}{(\roman{enumi})}

% BITTE NICHT ÄNDERN: interne Kommandos für die Informationen (Blattnummer, Gruppe, ...)
% PLEASE DO NOT EDIT THIS SECTION

\newcommand{\printsheet}{?}
\newcommand{\sheet}[1]{%
\renewcommand{\printsheet}{#1}%
}

\newcommand{\printgroup}{?}
\newcommand{\group}[1]{%
\renewcommand{\printgroup}{#1}%
}

\newcommand{\printmembers}{}
\newcommand{\printmember}[3]{{#2} {#3} & {#1} \\}

\newcommand{\pdfmembers}{}
\newcommand{\pdfmember}[3]{{#1} {#3}, {#2}; }

\newcommand{\member}[3]{%
\expandafter\renewcommand\expandafter\printmembers\expandafter{\printmembers\printmember{#1}{#2}{#3}}%
\expandafter\renewcommand\expandafter\pdfmembers\expandafter{\pdfmembers\pdfmember{#1}{#2}{#3}}%
}

\AtBeginDocument{\hypersetup{
    pdftitle = {Übungsblatt \printsheet},
    pdfauthor = {Abgabegruppe \printgroup: \pdfmembers},
    pdfsubject = {Mathematische Logik}
}}

% <-------- HIER alle Informationen eintragen ========================
% enter your information here
\sheet{10} % Nummer des Blatts / number of exercise sheet
\group{55} % Gruppennummer der Abgabegruppe in Moodle / group number from Moodle
% Die Gruppennummer erscheint NICHT auf dem Blatt (nur in den PDF-Metadaten).
% The group number does NOT appear on the sheet (check the PDF meta data).

% alle Gruppenmitglieder in der Form \member{Matrikelnummer}{Vorname}{Nachname}
% group members are entered as \member{matriculation number}{first name}{last name}
\member{405401}{Marc}{Ludevid}
\member{405409}{Andrés}{Montoya}
\member{405959}{Til}{Mohr}
%\member{999999}{Viertes}{Mitglied}

\begin{document}

% Platz für die Punktetabelle und Kommentare
\hfill
\begin{Form}
\begin{tabular}{c}
\\
Gesamtpunkte: \\[2mm]
\TextField[name=points, width=20mm, align=1, bordercolor={0 0 0}]{} \\
\\
\end{tabular}
\end{Form}

% Kopfzeile
{\raggedright
\begin{tabular}{l}
    MaLo \\
    SS 2021 \\
    \today{} \\
\end{tabular}}
\hfill
{\Large Übungsblatt \printsheet}
\hfill
\begin{tabular}{l l}
\printmembers
\end{tabular}
\hrule


% <-------- HIER beginnt die Lösung ========================
% your SOLUTION starts here

\section*{Aufgabe 1}
E-Test

\section*{Aufgabe 2}
\begin{enumerate}[label=(\alph*)]
	\item	$R^{\mathfrak{h}(T)} = \{c_0, f^2c_0\}$
	\item	Nein. $\mathfrak{h}(T)$ ist ein Modell von $f^4c_0 \neq c_1$, jedoch ist $f^4c_0 = c_1 \in T$. (Skript Seite 99)
	\item	Beobachtungen:
			\begin{itemize}
				\item	Wegen $fc_0 = fc_1$ gilt für alle $n \in \N, n \geq 1$: $f^nc_0 = f^nc_1$
				\item	Da $f^4c_0 = c_1$ gilt auch $f^4c_1 = c_1$. Es gilt also $f^4c_0 \xmapsto{f} c_1$ und $f^4c_1 \xmapsto{f} c_1$.
				\item	Es gilt auch $Rf^2c_1$ und für alle $n \in N$: $Rf^{2 + 4 \cdot n}c_0, Rf^{2 + 4 \cdot n}c_0$.
			\end{itemize}
			Sei $G_\tau$ die Menge aller Grundterme über $\tau$.
			\begin{align*}
				\Sigma &= \{t = t' \mid t,t' \in G_\tau, \text{ in } t \text{ bzw. } t' \text{ kommen } k \geq 1 \text{ bzw. } m \geq 1 \text{ } f \text{ vor, mit } k = 4 \cdot z \cdot m, z \in \Z\} \\
					&\cup \{t = c_1, c_1 = t \mid t \in G_\tau, \text{ in } t \text{ kommt } f \text{ } k \text{-mal vor, mit } k = 4 \cdot (1 + n), n \in \N\} \\
					&\cup \{c_0 = c_0, c_1 = c_1\} \\
					&\cup \{Rf^kc_0, Rf^kc_1 \mid k = 2 + 4 \cdot n, n \in \N\} \\
					&\cup \{Rc_0\}
			\end{align*}
	\item	Nein. $\mathfrak{h}(T)$ ist kein Modell von $Rf^2c_1$, $\mathfrak{h}(\Sigma)$ jedoch schon.
	\item	\begin{align*}
				[c_0]_\sim &= \{c_0\} \\
				[c_1]_\sim &= \{c_1, f^kc_0, f^kc_1 \mid k = 4 \cdot (n + 1), n \in \N\} \\
				[fc_0]_\sim &= \{f^kc_0, f^kc_1 \mid k = 4 \cdot n + 1, n \in \N\} \\
				[f^2c_0]_\sim &= \{f^kc_0, f^kc_1 \mid k = 4 \cdot n + 2, n \in \N\} \\
				[f^3c_0]_\sim &= \{f^kc_0, f^kc_1 \mid k = 4 \cdot n + 3, n \in \N\} \\
				\\
				R^{\mathfrak{A}(\Sigma)} &= \{[c_0]_\sim, [f^2c_0]_\sim\} \\
				\\
				f:& [c_0]_\sim \mapsto [fc_0]_\sim & [c_1]_\sim \mapsto [fc_0]_\sim \\
				& [fc_0]_\sim \mapsto [f^2c_0]_\sim & [f^2c_0]_\sim \mapsto [f^3c_0]_\sim \\
				& [f^3c_0]_\sim \mapsto [c_1]_\sim
			\end{align*}
\end{enumerate}


\newpage


\section*{Aufgabe 3}
\begin{enumerate}[label=(\alph*)]
	\item	\begin{enumerate}[label=(\roman*)]
				\item	Wenn es ein $\psi \in \operatorname{Th}(\A) \cap \overline{\operatorname{Th}}(\A)$ gäbe, dann würde ja $\A \models \psi$ und $\A \not\models \psi$ gelten. Dies ist ein Widerspruch. Folglich sind $\operatorname{Th}(\A)$ und $\overline{\operatorname{Th}}(\A)$ disjunkt.
				\item	i) $\Gamma$ enthält die Gleichung $t = t$. Offensichtlich, denn $t = t$ ist in der Theorie der Struktur.\\
						ii) Angenommen $t = t'$ und $\psi(t)$ sind in der Theorie der Struktur. Dann ist auch $\psi(t')$ in der Theorie denn es gilt: $t = t', \psi(t) \models \psi(t')$. 	
				\item	Sei $\neg \psi \in \operatorname{Th}(\A)$. Es gilt folglich $\A \models \psi$. Angenommen es gilt aber $\psi \not\in \operatorname{Th}(\A)$. Da $\operatorname{Th}(\A) \cup \overline{\operatorname{Th}}(\A) = \operatorname{FO}(\tau)$ und beide disjunkt sind, muss $\psi \in \operatorname{Th}(\A)$, also $\A \models \psi$. Da $\A$ aber beliebig ist, ist dies ein Widerspruch. Folglich muss $\psi \in \overline{\operatorname{Th}}(\A)$ \\
						Analog umgekehrt.
				\item	Sei $\psi \lor \theta \in \operatorname{Th}(\A)$. Dann gilt $\A \models \psi \lor \theta$. $\A$ ist also ein Modell von mindestens eins von beidem. Es muss also $\A \models \psi$ oder $\A \models \theta$ gelten. Folglich muss auch eines zu $\operatorname{Th}(\A)$ gehören. \\
						Sei $\psi \lor \theta \in \overline{\operatorname{Th}}(\A)$. Dann gilt ja $\A \not\models \psi \lor \theta$. Ist nun einer der beiden Formeln nicht in $\overline{\operatorname{Th}}(\A)$, so muss diese Formel dann in $\operatorname{Th}(\A)$ sein. Dann ist aber $\A$ ein Modell der Formel, weswegen $\A \models \psi \lor \theta$ gelten müssten. Widerspruch. Folglich muss $\psi, \theta \in \overline{\operatorname{Th}}(\A)$ gelten.
			\end{enumerate}
	\item	Wenn $\psi \land \theta \in \Gamma^*$, dann gehören $\psi$ und $\theta$ zu $\Gamma^*$. Wenn $\psi \land \theta \in \Delta^*$, dann gehört $\psi$ oder $\theta$ zu $\Delta^*$.
	\item	Wenn $\psi \impl \theta \in \Gamma^*$, dann gehört $\psi$ zu $\Delta^*$ oder $\theta$ zu $\Gamma^*$. Wenn $\psi \impl \theta \in \Delta^*$, dann gehören $\psi$ zu $\Gamma^*$ und $\theta$ zu $\Delta^*$. \\
			
			Sei $\psi \impl \theta \in \operatorname{Th}(\A)$. Dann ist entweder $\A$ kein Modell von $\psi$. Also gilt $\A \not\models \psi$, weshalb $\psi \in \overline{\operatorname{Th}}(\A)$. Oder $\A$ ist Modell von sowohl $\psi$ als auch $\theta$. Dann ist folglich $\theta \in \operatorname{Th}(\A)$, da ja $\A \models \theta$.\\
			Sei $\psi \impl \theta \in \overline{\operatorname{Th}}(\A)$. Dann ist ja $\A$ zwar ein Modell von $\psi$, aber keines von $\theta$. Deshalb gilt $\A \models \psi$ und $\A \not\models \theta$, also ist $\psi \in \operatorname{Th}(\A)$, aber $\psi \in \overline{\operatorname{Th}}(\A)$.
	\item	Offensichtlich ist $\exists x \psi(x) \in \operatorname{Th}(\A)$ mit $\psi \coloneqq 2 = x \cdot x$ ($x \in \R, x = \sqrt{2}$). Jedoch sind alle Grundterme rationale Zahlen. Es gibt also keinen Grundterm $t$, bei dem $\A \models \psi(t)$. Folglich ist $\psi(t) \not\in \operatorname{Th}(\A)$.
\end{enumerate}


\newpage


\section*{Aufgabe 4}
\begin{enumerate}[label=(\alph*)]
	\item	
	\item	
	\item	Falsch! \\
			Seien $\tau \coloneqq \{<\}$, $\Phi \coloneqq \{\exists x \forall y (x = y \lor x < y), \exists x \forall (x = y \lor y < x)\}$. \\
			 $\Phi$ axiomatisiert also die Klasse der $\tau$-Strukturen mit kleinstem und größtem Element. $\operatorname{Mod}(\phi_1)$ axiomatisiert nur die Klasse der $\tau$-Strukturen mit kleinstem Element, $\operatorname{Mod}(\phi_2)$ axiomatisiert nur die Klasse der $\tau$-Strukturen mit größtem Element. \\
			Widerspruch!
	\item	Gegenbeispiel: Die Klasse der Endlichen Linearen Ordnungen welche endlich axiomatisierbar ist durch $\phi = \forall x (\forall y (y < x \lor y = x) \lor \exists y (x < y \land \neg \exists z (x < z < y))) \land \exists x (\forall y (x < y \lor x = y)$\\
			Also gibt es ein kleinstes und grösstes Element und jedes Element hat einen eindeutigen Nachfolger.\\
			Aber für das unendliche Axiomensystem $\Psi$ gibt es keine endliche Teilmenge, die die Klasse axiomatisiert und somit auch keine Konjunktion von Sätzen in $\Psi$, die die Klasse axiomatisiert.\\
			Sei $\psi_n = \exists x_1 \exists x_2 \dots \exists x_n ( x_1 \not = x_2 \land x_1 \not = x_3 \land \dots \land x_1 \not = x_n \land x_2 \not = x_3 \land\dots \land x_{n-1} \not = x_n \land \forall y ((y = x_1 \lor y = x_2 \lor \dots \lor y = x_n) \land x_1 < x_2 \land x_2 < x_3 \land \dots \land x_{n-1} < x_n)$

			$\psi_n$ axiomatisiert die Lineare Ordnung mit n Elementen. Also ist $\Psi = \{\psi_n : n \in \mathbb{N}\}$ ein unendliches Axiomensystem für die Endlichen Linearen Ordnungen.
		Jegliche endliche Untermenge von $\Psi$ hat ein $\psi_m$ für maximales $m$. Somit kann eine Lineare Ordnung gewählt werden mit $m+1$ Elementen und diese wird kein Modell von besagten Untermenge sein.
	\item	Richtig! \\
			Es gibt ein endliches Axiomensystem $\Phi'$ für $K$. $K$ ist also auch axiomatisierbar durch $\phi \coloneqq \bigwedge \Phi'$. Für jede Struktur $\A \in K$ muss also gelten $\A \models \phi$ und für jede Struktur $\B \not\in K$ ($\B \in \overline{K}$) $\B \not\models \phi$. Folglich muss für $\B$ dann gelten $\B \models \overline{\phi}$ mit $\overline{\phi} \coloneqq \neg \phi = \neg \bigwedge \Phi = \bigvee\limits_{\phi' \in \Phi} \neg \phi'$. Folglich ist $\{\overline{\phi}\}$ ein endliches Axiomensystem von $\overline{K}$.
\end{enumerate}


\end{document}
